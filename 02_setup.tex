\section{Setup}

% Siamese GCN.
% Aggregate
% Set up cost matrix (bp vs. normal, scale with $\alpha$).
% norms
% Compoute explicit matching with sinkhorn (use cost matrix as kernel, can tune $\lambda$).
% Use optimal matching for backprop.
% Multiply matching with cost matrix.

For our experiments we use the a siamese network structure with multiple shared message passing layers, followed by the matching step.


Let $d_h$ be an arbitrary embedding size, $d_{\text{node}}$ be the number of node features of the input graphs, $d_{\text{edge}}$ be the number of edge features, $W_l \in \mathbb{R}^{d_h \times d_h}$, $b_l \in   \mathbb{R}^{d_h}$be a weight matrix and bias, $h_i^l \in \mathbb{R}^{d_h}$ be the embedding of node $i$ at layer $l$ for $l > 0$, $\sigma(\cdot)$ be an elementwise non-linearity, $E \in \mathbb{R}^{d_{\text{edge}} \times d_h \times d_h}$ be a weight tensor ($h_i^0 = v_i \in \mathbb{R}^{d_{\text{node}}}$ and $W_0 \in \mathbb{R}^{d_{\text{node}} \times d_h}$).

In our first experiment we use the following message passing function:
\begin{equation}
     h_i^{l} = \sigma(W_{l} h_i^{l-1} + b_l) + \sum_{j \rightarrow i} \sigma(W_{l} h_j^{l-1} + b_l) E_{e_{j \rightarrow i}}
\end{equation}
where $E_{e_{j \rightarrow i}} \in \mathbb{R}^{d_h \times d_h}$ is the weight tensor indexed by the discrete feature of the edge connecting node $j$ with node $i$.

To compute the entries of the cost matrix we concatenate all node embeddings and take the norm of pairwise distances across both graphs, or the embeddings scaled by a trainable $\alpha \in \mathbb{R}^{d_\text{final}}$:
\begin{equation}
     \begin{gathered}
          h_i^\text{final} = [h_L^i ; \dots ; h_1^i ; \sigma(W_{1} h_i^{0} + b_0)] \in \mathbb{R}^{d_\text{final}}\\
          C_{i,j} = \norm{h_i^\text{final} - h_j^\text{final}} \quad
          C_{i, \epsilon} = \norm{ \alpha h_i^\text{final}} \quad
          C_{\epsilon, j} = \norm{ \alpha h_j^\text{final}} \quad i \in \{1 \dots N_1\}, j \in \{1 \dots N_2\}
     \end{gathered}
\end{equation}

In our experiment we tested three different norms: the euclidian norm ($p=2$), the manhattan norm ($p=1$), and simple two layer perceptron (MLP). The first layer does not change the embedding size, then we apply a relu non-linearity. The second layer reduces the embedding size to a single number to which we apply the softplus function ($\ln(1 + e^x)$).

As for the matrix $C$ itself we tested two variants. The first one we call BP-Distance on account of bipartite graph matching. Due to implemenation advantages we decided to padd the pairwise distances with rows and columns of $C_{i, \epsilon}, C_{\epsilon, j}$ respectively, instead of the $\infty$-values typically used.
\begin{equation}
     C_\text{BP}=
          \left[
          \begin{array}{ccc|ccc}
               C_{1,1} & \dotsi & C_{1, N_2} & C_{1, \epsilon} & \dotsi & C_{1, \epsilon} \\
               \vdots & \ddots & \vdots & \vdots & \ddots & \vdots \\
               C_{N_1, 1} & \dotsi & C_{N_1, N_2} & C_{N_1, \epsilon} & \dotsi & C_{N_1, \epsilon} \\
               \hline
               C_{\epsilon, 1} & \dotsi & C_{\epsilon, N_2} & 0 & \dotsi & 0 \\
               \vdots & \ddots & \vdots & \vdots & \ddots & \vdots \\
               C_{\epsilon, 1} & \dotsi & C_{\epsilon, N_2} & 0 & \dotsi & 0 \\
          \end{array}
          \right]
     \in \mathbb{R}^{(N_1 + N_2) \times (N_1 + N_2)}
\end{equation}
This does lead to different results after sinkhorn normalization due to the entropy regularization, but the network can easly apdapt by increasing $\norm{\alpha}_2$. Therefore this modification should not effect our trainable model. The second variant for the cost matrix is simply the smallest square sub-matrix of the one above that contains all pairwise distances, such that $C_\text{no-BP} \in \mathbb{R}^{\max({N_1, N_2}) \times \max({N_1, N_2})}$.

The final distance $d$ is then computed and trained via the mean squared error between target distance and prediction:
\begin{equation}
     \begin{gathered}
          M = \text{sinkhorn-normalization}(e^{-\frac{C}{\lambda}}) \\
          d = \norm{M * C}_F \\
          \text{loss} = (d - d_\text{target})^2
     \end{gathered}
\end{equation}

\cite{fey2020_update} have recently shown that a very similar model to ours can be used to learn graph matchings. The main difference to our model is that they normalize the cost matrix $C$ instead of kernel matrix $e^{-\frac{C}{\lambda}}$. However, they report bad gradients with sinkhorn normalization and therefore use row-wise sofmax instead.


They only use the left upper part of the cost matrix (not a problem because aim for hits@1)
and they use the dot product of node embeddings instead of a p-norm or MLP.

We use p=2, and no-BP, spline-CNN, scale the kernel matrix by $sum(C) / max(n1,n2)^2$
Now we need to backprop through the sinkhorn iteratons!

We also implemented a consensus step based on their work. We use instead of
\\
$C_0 = h_s \cdot h_t$\\
$C_0 = \norm{h_s - h_t}_2$ fill with norms\\
$C_l = C_{l - 1} + \text{MLP}(h_s - h_t)$\\
$C_l = \beta_l C_0 + (1 - \beta_l) \text{MLP}(h_s - h_t)$ keep old norms scaled by $\alpha^2$\\
$M_l = \text{softmax}(C_l)$ \\
$M_l = \text{sinkhorn}(e^{-\frac{C_l * \text{reg}}{\lambda}}) \quad \text{reg} = \frac{\norm{C_l}_1}{\max(n1,n2)^2}$\\


Therefore we set up a second experiment to show, that applying the sinkhorn normalization to the kernel matrix instead of the



Implementation. We can run everything in batches, either with padded matricies (if all graphs are close to max-nodes) or in COO sparse, allowing us to batch larger graphs together with smaller ones.
