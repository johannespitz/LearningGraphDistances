\section{Introduction}

% Graphs are intersting because you can represent anything as a graph.
% Convolutional networks were extremly successful \cite{alexnet2012}.
% Graph Convolution Network, Kipf and Welling \cite{kipf2017}.
% Most research involving deep learning and graphs has focused on node and graph classification, but recently researchers have started applying these tools to the problem of graph distance estimation.

Most conceivable datasets can be represented as a graph or a set of graphs. Not only “graph-like” datasets such as social networks or chemical molecules fall into this category, also, i.e. executable binaries can be represented by control flow graphs or pictures can be represented by the graph of keypoints. % write better in more sentences

In this report we investigate the two problems:

graph matching . A graph matching establishes a node correspondence between graphs, maximising corrosponding node and edge affinity (\citealp{fey2020_update}; \citealp{wang2019}). For a detailed review of classical research into graph matching, recent neural network methods, and the applications of graph matching we refer the reader to \cite{fey2020_update}.
% Interesting applications here refer to Fey later maybe right before that paragraph

and graph distance (or similarity) estimation.
Graph distance or grpah similarity estimation refers to the problem of learning a metric relating different graphs from a given distribution. Intersting applications include detecting fraudulent binaries by means of control flow graphs (\citealp{{li2019}}), graph-based keyword spotting {\citealp{riba2018}}, and in particular predicting the Graph Edit Distance.
% but want to be able to learn any metric

The GED is widly used in graph similarity search, i.e. with reduced graphs of chemical compounds (\citealp{chem2006}), in fingerprint matching (\citealp{fingerprint2005}), or image indexing (\citealp{image_index2008}). Since computing the GED is NP-complete \cite{np_complete1998} it usually needs to be approximated.

Since computing the GED is NP-complete \cite{np_complete1998} it usually needs to be approximated. There are multiple classical algorithms to approximate the GED, some of them can gurantuee to find a lower bound (\cite{hungarian2009}) or upper bound (\citealp{hed2015}). However, due to their origin in optimization there is always a trade off between speed and accuracy, in which the faster approximations consider only very local node structures (\citealp{riba2018}).



Previoius attemps of using neural networks for this have the short come that they cannot scale to larger graphs due to bottleneks (embedding vectors), or they (Riba) don't use Danskin.

We present a fast method to one using the other.

Main contributions:
\begin{itemize}
    \item Propose method for graph distance learing that scales to larger graphs. \\ (With Nystrom/Multiscale $\rightarrow$ First sub $n^2$ that scales to larger graphs).
    \item Experimentaly show that it outperforms other methods on task of Graph Edit Distance estimation.
    \item Show that our application of sinkhorn has advantages over previous work.
\end{itemize}
