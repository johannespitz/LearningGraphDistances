\section{Introduction}

Most conceivable datasets can be represented as a graph or a set of graphs. Not only “graph-like” datasets such as social networks or chemical molecules fall into this category, also, i.e. executable binaries can be represented by control flow graphs or pictures can be represented by the graph of keypoints. % write better in more sentences
The broad applicability of graphs, and the success of deep learning across different tasks [vision, rl, nlp] has spurred on research in the application of neural networks to graphs. Graph Neural Networks (GNN), based on message passing between connected nodes, have already showed great success in predicting properties of nodes (\cite{kipf2017}) and of entire graphs (\citealp{gilmer2017}). However, only very recently researchers started applying GNN's to tasks such as graph distance or similarity estimation and graph matching.

\textbf{Graph Distance/Similarity.} A metric or a distance function defines a distance between paris of elements of a set, and similarity is the inverse of a distance. Finding similar graphs can be crucial for fingerprint matching (\citealp{fingerprint2005}), image indexing (\citealp{image_index2008}), or while working with chemical compounds (\citealp{chem2006}). Often the graph edit distance (GED) is used a proxy for finding similar graphs. Since computing the GED is NP-complete (\citealp{np_complete1998}) it usually needs to be approximated. There are multiple classical approximating algorithms, some of them can gurantuee to find a lower bound (\citealp{hungarian2009}) or upper bound (\citealp{hed2015}). However, due to their origin in optimization there is always a trade off between speed and accuracy, in which the faster approximations consider only very local node structures (\citealp{hungarian2009}). Recently, trained GNN's have been used to improve upon classical algorithms in terms of accuracy and inference times (\citealp{bai2019}). %\cite{bai2019} showed that trained GNN's can be used instead of classical algorithms to estimate the GED with much shorter inference times.
One key advantage of a neural network models is that they are not limited to learning the GED, but they can also directly learn a given similarity function. Interesting examples include detecting fraudulent binaries (\citealp{li2019}), or graph-based keyword spotting ({\citealp{riba2018}).

\textbf{Graph Matching.} A graph matching establishes a node correspondence between graphs, maximizing corresponding node and edge affinity (\citealp{wang2019}). Matching two graphs can be interesting for linking user accounts across different social network graphs (\citealp{zhang2016}), tracking keypoints of moving objects (\citealp{vento2012}), or aligning protein networks in bioinformatics (\citealp{singh2008}).

Previously designed neural network models for graph distance estimation either reduce the entire graph to a single embedding vector creating a bottleneck if scaling to larger graphs (\citealp{bai2019}, \citealp{li2019}), or they implicitly use a crude and strict matching (\citealp{riba2018}).

We propose a model that can learn arbitrary metrics on graphs using a novel cost matrix and explicit soft matchings, similar to that used by \cite{fey2020_update} in the context of graph matching. The main contributions of this report are:
\begin{itemize}
    \itemsep0em
    \item The Graph Distance Network (GDN): A model derived from principle properties of metrics and classical algorithms for approximating the GED that uses a novel cost matrix soft matchings and fast backpropagation.
    \item Experiments showing that our model improves upon state-of-the-art neural networks in predicting the GED.
    \item Empirical evidence that our novel cost matrix and different application of the sinkhorn normalization improve upon state-of-the-art results of \cite{fey2020_update} in graph matching.
\end{itemize}


% Faced with the great significance yet huge difficulty of computing the exact GED between two graphs,
% a flurry of approximate algorithms have been proposed with a trade-off between speed and accuracy.
% However, these methods usually require rather complicated design and implementation based on discrete optimization or combinatorial search. The time complexity is usually polynomial or even sub-exponential in
% the number of nodes in the graphs, such as HED
% (Fischer et al. 2015), Hungarian (Riesen and Bunke 2009),
% VJ (Fankhauser, Riesen, and Bunke 2011), A*-Beamsearch
% (Beam) (Neuhaus, Riesen, and Bunke 2006), etc



% TODO
% For a detailed review of classical research into graph matching, recent neural network methods, and the applications of graph matching we refer the reader to \cite{fey2020_update}.
