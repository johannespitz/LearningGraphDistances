\appendix
\appendixpage


\begin{table}[htbp]
    \addtolength{\tabcolsep}{-1pt}
    \fontsize{9pt}{10.25pt}\selectfont
    \centering
    \renewcommand{\arraystretch}{1.2}
    \begin{tabular}{|l|c|c|c|c|c|c|c|}
        \hline
        Dataset & Node Feat & Edge Feat & \#Train & \#Val/Test & Max Nodes & Avg Nodes & Avg Edges \\
        \hline
        Pref-Att & 6 & 4 & 144 & 48 & 30 & 20.6 & 75.4 \\ % edges mean 75.4
        \hline
        Aids & 53 & 4 & 144 & 48 & 30 & 20.6 & 44.6 \\ % edges mean 44.6
        \hline
    \end{tabular}
    \caption{Datasets}
    \label{tab:ex1-data}
\end{table}

Note \#Train and \#Val/Test are the number of graphs. We train and validate/test on all possible pairs of graphs, except the pairs where a graph is paired with itself.

Note: For a real paper the related work section would look different of course
Probably it would then be possible to move that section before the derivation, but it is there, because there is some background information given in the derivation that is required for such a detailed description of the related models.
since this is a report and I spent a significant amout of time working with these baselines, I think the level of detail is approriate.


In this report I summarize the work I did during my guided research. The main part of the report is focuses on the findings that are relevant to a potential paper submission, while in the appendix I will write more about the process and some of the less important work I have done.

The main part of the report starts with an analysis of related work in the area of graph distance learning and graph matching. Then \dots


Search for datasets

\section{Yeast Dataset}

\section{Control Flow Dataset}
