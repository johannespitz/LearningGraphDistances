\appendix
\appendixpage

In this report I summarize the work I did during my guided research. The main part of the report focuses on findings that are relevant to a potential paper submission, but here in the appendix I will add some information about the datasets we used, and some we did not use (for completeness).


\section{Graph Edit Distance Datasets}
\begin{table}[htbp]
    \addtolength{\tabcolsep}{-1pt}
    \fontsize{9pt}{10.25pt}\selectfont
    \centering
    \renewcommand{\arraystretch}{1.2}
    \begin{tabular}{|l|c|c|c|c|c|c|c|}
        \hline
        Dataset & Node Feat & Edge Feat & \#Train & \#Val/Test & Max Nodes & Avg Nodes & Avg Edges \\
        \hline
        Pref-Att & 6 & 4 & 144 & 48 & 30 & 20.6 & 75.4 \\ % edges mean 75.4
        \hline
        Aids & 53 & 4 & 144 & 48 & 30 & 20.6 & 44.6 \\ % edges mean 44.6
        \hline
    \end{tabular}
    \caption{Datasets}
    \label{tab:ex1-data}
\end{table}

Note \#Train and \#Val/Test are the number of graphs. We train and validate/test on all possible pairs of graphs, except the pairs where a graph is paired with itself.

\section{Yeast Dataset}

The Yeast Dataset (\href{https://www3.nd.edu/~cone/MAGNA++/}{link}), used by \cite{yeast2019}, contains the protein-protein interaction (PPI) network of yeast and noise versions thereof. The base network contains 1,004 proteins and 4,920 high-confidence interactions. The noisy versions contain additional low-confidence interactions.

The first obstacle with this dataset would have been the generation of train/val/test splits.  Moreover, we could not have compared our results directly with \cite{yeast2019} because they learn unsupervised. The second obstacle is that the dataset contains no node or edge features. That is mostly a problem because GNN's are particularly good at using node, and also edge features. Therefore, it is questionable if a  model based on GNN's would perform impressive enough. In the end we decided not to use this dataset, also because we discovered the paper by \cite{fey2020_update} who use plenty of graph matching datasets that fit much better to our model.

\section{Control Flow Dataset}

Code to generate the splits: \href{https://github.com/johannespitz/functionsimsearch/tree/master}{github}

At the time of writing the datasets can be found at '/nfs/students/'
