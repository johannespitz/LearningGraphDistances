\section{Outlook}

The first problem that should be tackled is the gap between validation and test results on the AIDS dataset. Increasing the number of graphs could potentially solve the issue already. However, to verify any improvement we should also consider working with multiple completely distinct validation sets at once.

Clearly, also the underperforming consensus method should be improved in the future. We are able to produce a solid initial matching, and simply applying the same "softmax-no-norms" consensus of \cite{fey2020_update} should yield at least as good results as they report. Therefore, exactly replicating their model might be a good start and allow us to then investigate which modifications can further improve these results.

After that we might want to look back the four defining properties of metrics:
\begin{itemize}
    \itemsep0em
    \item $d(x,y) \ge 0$ non-negativity
    \item $d(x,y) = 0 \Leftrightarrow x = y$ identity of indiscernibles
    \item $d(x,y)  = d(y,x)$ symmetry
    \item $d(x,y) \le d(x,z) + d(z, y)$ subadditivity or triangle inequality
\end{itemize}
We already ensured non-negativity, and symmetry, but it would certainly be interesting to investigate if we can ensure the triangle inequality. Perhaps that could lead to a small modification which induces a strong and useful prior. Also investigating the identity of indiscernibles could lead to interesting results. If it was possible to ensure this property, it would solve the graph isomorphism problem, and finally answer the question of whether the problem is NP-complete or not. However, doing so will require the use of more powerful embedding networks because a standard GNN cannot distinguish between certain graphs\footnote{For example, a GNN cannot distinguish between a hexagon and two disconnected triangles.} and will therefore yield exactly the same set of embeddings in such a case. Maybe easier to handle and therefore more promising is translational invariance, $d(x,y) = d(x+a,y+a)$, which might be a valid assumptions in many cases. Graph sums can be defined by adding the adjacency matrices of equal sized graphs \cite{graph_sum2004} or, in the context of the graph edit distance, maybe by adding a node that is connected to all other nodes. If it is not possible to derive enforcing model modifications for translational invariance or the triangle inequality then one could still try using the properties for data augmentation, which could certainly lead to better generalization to the test set.

A different direction for further research is to speed up the current model. It seems inevitable that the whole process is quadratic in the number of nodes due to the use of the cost matrix $C \in \mathbb{R}^{N \times N}$. However, we might be able to achieve linear run time by using the Nystr{\"{o}}m method or multiscale matching. \cite{nytrom2019} show that using the Nystr{\"{o}}m method and Sinkhorn scaling one can accurately approximate the Sinkhorn distance without ever computing the full cost matrix. The Nystr{\"{o}}m method is based on a set of landmarks that is then used to speed up matching. Similarly multiscale matching uses clusters of nodes that are then matched with each other (\citealp{multiscale2016}). One problem with both approaches is that they only produce the final distance and not an explicit matching or a complete cost matrix, which would be required to apply Danskin's theorem for fast backpropagation through the matching. Therefore, the key for implementing both of these methods will be to find an efficient way to backpropagate the gradients. Ideally one would combine both approaches since landmarks and clusters each have their merits and demerits. Using landmarks one can vastly overestimate small distances, and multiscale matching never computes distances which exceed some threshold.

To show off the advantage of our full BP cost matrix it would be interesting to investigate "sparse matchings". \cite{fey2020_update} use only pairs of graphs where every node in the source graph finds a match in the target graph. We could simply add the other pairs to the training set, but validate on both groups individually. There should be a good chance that we significantly outperform any previous methods on this specialized task.
