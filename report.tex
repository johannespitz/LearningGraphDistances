\documentclass[a4paper,10pt]{article}
\usepackage[utf8]{inputenc}
\usepackage[margin=3cm]{geometry}
\usepackage{amsmath}
\usepackage{amssymb}
\usepackage{amsthm}
\usepackage{fancyhdr}
\usepackage{seminar}
\usepackage{graphicx}
% \usepackage{subfigure}
\usepackage{float}
\usepackage{hyperref}

\usepackage[round]{natbib}   % omit 'round' option if you prefer square brackets

\usepackage{hhline}
\usepackage{multirow}
\usepackage{array}
\newcommand{\PreserveBackslash}[1]{\let\temp=\\#1\let\\=\temp}
\newcolumntype{C}[1]{>{\PreserveBackslash\centering}p{#1}}

\usepackage{mathrsfs}
\usepackage{commath}
\usepackage{mathtools}
\usepackage{subcaption}
\usepackage{appendix}

\DeclareMathOperator*{\argmax}{arg\,max}
\DeclareMathOperator*{\argmin}{arg\,min}

\pagestyle{fancy}


%You can add theorem-like environments (e.g. remark, definition, ...) if you want
\newtheorem{theorem}{Theorem}

\title{Learning Graph Distances} % Replace with your title
% \newcommand{\shorttitle}{\title}
% \shorttitle{Learning graph distances}
\author{Johannes Pitz} % Replace with your name
\institute{\textit{Guided Research: Data Analytics and Machine Learning Group  \protect\\ TUM Department of Informatics}}

\makeatletter
\let\runauthor\@author
\let\runtitle\@title
\makeatother
\lhead{\runauthor}
\rhead{\runtitle}


\begin{document}

\title{Learning Graph Distances \protect\\ via Graph Neural Networks and Node Matching}
\maketitle

\begin{abstract}


Computing meaningful distances between graphs is often difficult due to the combinatorial explosion of possible transformations on graphs. Recently, researchers started using graph neural networks in an attempt to replace classical search algorithms with learning methods. However, these learning methods often rely on fixed-size graph embeddings, which prevent them from scaling to larger graphs. We propose to use soft matchings between nodes to overcome this problem. Our novel combination of a classical cost matrix, motivated by bipartite graph matching algorithms, and the modern Sinkhorn distance shows state-of-the-art results for predicting the ubiquitous graph edit distance. Additionally, we report strong empirical results at graph matching with the same model. Our implementation\footnote{\url{https://gitlab.lrz.de/gdn/graph-distance}} efficiently handles sparse inputs and large real-world graphs.


\end{abstract}

\section{Introduction}

% Graphs are intersting because you can represent anything as a graph.
% Convolutional networks were extremly successful \cite{alexnet2012}.
% Graph Convolution Network, Kipf and Welling \cite{kipf2017}.
% Most research involving deep learning and graphs has focused on node and graph classification, but recently researchers have started applying these tools to the problem of graph distance estimation.

Most conceivable datasets can be represented as a graph or a set of graphs. Not only “graph-like” datasets such as social networks or chemical molecules fall into this category, also, i.e. executable binaries can be represented by control flow graphs or pictures can be represented by the graph of keypoints. % write better in more sentences

The broad range of applicability (of graphs), and the recent success of deep learning across different tasks [vision, rl, nlp] has spurred on the research in the application of deep neural networks to graphs. Borowing the idea of convolutional layers (\citealp{alexnet2012}) from conputer vision, \cite{kipf2017} introduced Graph Convolutional Networks (GCN) which can be implemented efficiently via message passing between connected nodes. Based on this message passing mechanism researchers have improved upon classical state-of-the-art methods in node classification and graph classification [find something in the pytorch-geometric list].

In this report we investigate the two problems:

\textbf{Graph Matching.} A graph matching establishes a node correspondence between graphs, maximising corrosponding node and edge affinity (\citealp{fey2020_update}; \citealp{wang2019}). For a detailed review of classical research into graph matching, recent neural network methods, and the applications of graph matching we refer the reader to \cite{fey2020_update}.
% Interesting applications here refer to Fey later maybe right before that paragraph

\textbf{Graph Distance/Similarity.} Graph distance or grpah similarity estimation refers to the problem of learning a metric relating different graphs from a given distribution. Intersting applications include detecting fraudulent binaries by means of control flow graphs (\citealp{{li2019}}), graph-based keyword spotting {\citealp{riba2018}}, and in particular predicting the Graph Edit Distance.
% but want to be able to learn any metric

The GED is widly used in graph similarity search, i.e. with reduced graphs of chemical compounds (\citealp{chem2006}), in fingerprint matching (\citealp{fingerprint2005}), or image indexing (\citealp{image_index2008}). Since computing the GED is NP-complete \cite{np_complete1998} it usually needs to be approximated.

Since computing the GED is NP-complete \cite{np_complete1998} it usually needs to be approximated. There are multiple classical algorithms to approximate the GED, some of them can gurantuee to find a lower bound (\cite{hungarian2009}) or upper bound (\citealp{hed2015}). However, due to their origin in optimization there is always a trade off between speed and accuracy, in which the faster approximations consider only very local node structures (\citealp{riba2018}).

% Faced with the great significance yet huge difficulty of computing the exact GED between two graphs,
% a flurry of approximate algorithms have been proposed with a trade-off between speed and accuracy.
% However, these methods usually require rather complicated design and implementation based on discrete optimization or combinatorial search. The time complexity is usually polynomial or even sub-exponential in
% the number of nodes in the graphs, such as HED
% (Fischer et al. 2015), Hungarian (Riesen and Bunke 2009),
% VJ (Fankhauser, Riesen, and Bunke 2011), A*-Beamsearch
% (Beam) (Neuhaus, Riesen, and Bunke 2006), etc

Previoius attemps of using neural networks for this have the short come that they cannot scale to larger graphs due to bottleneks (embedding vectors), or they (Riba) don't use Danskin.

We present a fast method to one using the other.

Main contributions:
\begin{itemize}
    \itemsep0em
    \item Propose method for graph distance learing that scales to larger graphs. \\ (With Nystrom/Multiscale $\rightarrow$ First sub $n^2$ that scales to larger graphs).
    \item Experimentaly show that it outperforms other methods on task of Graph Edit Distance estimation.
    \item Show that our application of sinkhorn has advantages over previous work.
\end{itemize}


\section{Background}

Let $\mathcal{G}$ be a set of graphs, where each graph $G = \{V, E\} \in \mathcal{G}$ with nodes $V$ and edges $E$ is drawn from a common distribution. The nodes and edges may have arbitrary features. Let $N$ denote the number of nodes $\vert V \vert$. Usually we expect the number of edges $\vert E \vert$ to be much smaller than the number of all possible edges: $N^2$. Possible distributions include chemical molecules, control flow graphs, or synthetic preferential attachment graphs (\citealp{pref_att2002}).

Graph neural networks based on the message passing scheme (\citealp{gilmer2017}) aggregate the features of nodes in the neighborhood $\mathcal{N}_1$ of node $i$ along their edges with features $e_{j,i}$. The aggregate $a_i^l$ is then used to update the node feature $h_i^{l-1}$ in the next layer $l$.
\begin{equation}
    a_i^l = \operatorname{Aggregate}^l\left(\{\!\!\{(h_j^{l-1}, e_{j,i}) : j \in \mathcal{N}_1(i)\}\!\!\}\right), \quad
    h_i^l = \operatorname{Update}^l\left( h_i^{l-1}, a_i^l \right)
\end{equation}
Here $h_i^0 = v_i \in V$ and $\{\!\!\{ \dots \}\!\!\}$ denotes a multiset.

\section{Related Work}
\label{section:related_work}

Researchers have already applied graph neural networks to the task of GED approximation. Bai et al. proposed in 2018 (published \citealp{bai2019}) to apply the same network to both graphs and aggregate the resulting node embeddings with an attention layer into two global graph embeddings, which can be used to compute a distance. In addition to this fully differentiable and therefore trainable part, they compute a histogram of the pairwise distances between node embeddings to incorporate some of the local node level information. However, the histogram method is not differentiable and it is unclear if the model can properly exploit this information. Moreover, this process reduces any graph to fixed-size embeddings and is unlikely to scale to larger graphs without hand-tuning the embedding size. Despite these flaws the authors show that their method significantly outperforms classical methods while being orders of magnitude faster at inference.

Later \cite{bai2018_cnn1} proposed to apply, instead of a histogram, a standard 2-D Convolutional Neural Networks (CNN) to the matrix of pairwise distances. As the authors point out, the main problems with this approach are permutation invariance and spatial locality. Firstly, a CNN depends on the ordering of the nodes in the graph representation. Therefore, the model is not invariant to permutations. Secondly, CNN's introduce a strong structural prior by making the assumption that nearby points are strongly related while further apart points are not. This assumption increases the importance of node ordering further. The authors decided to use breadth-first-search node-ordering (\citealp{bfs2018}). They claim that nearby nodes are sorted close to each other. However, the ordering cannot be guaranteed to be unique. In fact, it is not yet known if there exists a polynomial algorithm to find a canonical ordering (\citealp{canonical2016}) and therefore the predictions can depend on the representation of the graphs. However, the authors show an interesting connection between convolutional kernels and the optimal assignment problem, and show strong empirical results compared to classical algorithms.

Independently, \cite{riba2018} proposed a similar model to estimate graph distances. They apply the same message passing network to both graphs generating sets $A$ and $B$ of node embeddings (one for each graph). The distance between two graphs is then computed as:
\begin{equation}
     d(G_{1}, G_{2}) = \frac{\sum_{a \in A} \inf_{b \in B} d(a, b) + \sum_{b \in B} \inf_{a \in A} d(a, b)}{N_1 + N_2}
\end{equation}
This equation is derived as a soft variant of the Hausdorff distance, which is also known as the Chamfer distance in computer vision (\citealp{chamfer1977}). Instead of approximating the GED directly they validate their method on graph classification and keyword spotting datasets. Their empirical results are therefore difficult to compare to Bai et al. (2018; 2019). This method is interesting because it generates a complete matching between nodes and might scale to larger graphs without hand-tuning the embedding size. Local node structures are incorporated during the message passing, and global node correspondence is used for predicting the final distance.

\cite{li2019} used a similar method, called Graph Matching Network (GMN), for graph classification and detection of vulnerable binaries. They search for vulnerable binaries by comparing the control flow graphs of functions across different compilers. The GMN also employs message passing layers to both graphs. However, they modify the message passing scheme to allow information of one graph to flow into the node updates of the other graph. % cross-graph attention formulas?
This process is fully differentiable and makes use of local node level and global graph level information but still suffers from the problem of reducing the entire graph to a single embedding vector which is expected to scale unfavorably to larger graphs. Moreover, computing the cross-graph matching vector costs $O(N_1 N_2)$. While all previously described models have technically the same complexity ($O(N_1 N_2 + \vert E_1 \vert + \vert E_2 \vert)$), the GMN computes this cross-graph matching vector for each layer.

% Note that in principle one could also apply GMN layers and then use the Soft Hausdorff Distance, making these two methods somewhat orthogonal.

\cite{fey2020_update} apply message passing networks to graph matching. In particular they test their method on keypoint matching of objects between pictures and on cross-lingual knowledge graph alignment. Although this task is different to those mentioned previously, their model is extremely similar to the one of \cite{riba2018}. They also apply the same network to both graphs and compute a matching. Instead of the Chamfer distance \cite{fey2020_update} use the Sinkhorn normalization \cite{sinkhorn2013} on the matrix of pairwise distances. The Sinkhorn normalization returns a doubly stochastic matrix, which is interpreted as a soft correspondence between nodes. One could immediately find a matching by taking the maximum likelihood estimate, but the authors then apply a newly proposed consensus step. They generate an injective node coloring, or random node embeddings in practice, for one graph and use the soft correspondence matrix to map these to the other graph. Using the generated node embeddings, they apply another message passing network to both graphs, yielding, after normalization, a new soft correspondence matrix. This step is repeated multiple times and the final answer is found from the last soft correspondence matrix. Unfortunately, the authors report that in practice they had to use the $\text{softmax}$ operator on each row, instead of Sinkhorn normalization due to lacking gradient information. However, when using $\text{softmax}$ the model is not necessarily symmetric anymore, i.e. $M(G_1, G_2) \neq M(G_2, G_1)$.

% This can be a problem if the graphs have different number of nodes.
% In a toy experiment they show that with consensus steps their method is robust to node additions and removal, but it certainly is less principled because the soft correspondence matrix and the resulting matching is not necessarily symmetric, $M(G_1, G_2) \neq M(G_2, G_1)$.

% consensus is orthogonal to our work


% maybe add how everyone computes their pairwise distances
% < a, b >, MLP(a - b), ||a - b||_p


% In the end Sinkhorn > soft hausdorff, and gmn is probably obsolete
% really fast only if we never generate a full matching


\section{Derivation of the model}

% (assume E << V^2)
Let $\mathcal{G}$ be a set of graphs, where each graph $G = \{V, E\} \in \mathcal{G}$ with nodes $V$ and edges $E$ is drawn from a common distribution. The nodes and edges may have arbitrary features. Usually we expect the number of edges $\vert E \vert$ to be much smaller than the number of nodes squared $\vert V \vert ^2$. Possible distributions include chemical molecules, control flow graphs, or synthetic preferential attachment graphs (\citealp{pref_att2002}).

While our model can in principle learn any metric on a set of graphs, we will focus on the ubiquitous GED for this derivation. The GED between two graphs is defined analog to the String Edit Distance.
\begin{equation}
     \text{GED}(G_{1},G_{2}) = \min_{(e_{1},...,e_{k}) \in \mathcal{P}(g_{1},g_{2})} \sum_{i=1}^{k} c(e_{i})
\end{equation}
where $e_{i}$ are edit operations: edge/node addition, removal, and substitution, $c(e)$ is the cost of an edit operation, and $\mathcal{P}(G_{1},G_{2})$ is the set of all possible edit paths that transform $G_{1}$ into a graph isomorphic to $G_{2}$.

In mathematics any metric is non-negative and symmetric ($d(x,y) = d(y,x)$). And the GED fulfills both these requirements, as long as the cost do. Therefore, a model for learning metrics should also be designed to be non-negative and symmetric. Moreover, the model should certainly be invariant to the representation of the graph, i.e. the ordering of the nodes in memory.
% TODO
% Note that the GED and any other proper metric on graphs is symmetric and invariant to the graph representation. Clearly the GED as defined above cannot depend on the ordering in which the nodes are saved in memory. Moreover, as long as the costs of substitution are symmetric, also the GED is symmetric ($\text{GED}(G_{1},G_{2}) = \text{GED}(G_{2},G_{1})$). Therefore, a neural network approximating the GED should also be symmetric and invariant to the graph representation.

A simple neural network model that fulfills all these requirements could embed both graphs into a common vector space and compute the distance between these two vectors. One can use message passing networks to generate node embeddings. Message passing can be applied to graphs with arbitrary features; it is invariant to the graph representation and common practice in machine learning on graphs due to efficient handling of sparse inputs. These node embeddings can then be aggregated into a single graph embedding, possibly using an attention layer. Finally, any distance between two vectors (cosine, $p$-norm, etc.) will ensure valid predictions. The main problem with this approach is that the embedding size becomes a bottleneck for larger graphs and the entire model would need to be trained from scratch to increase it.
%  \cite{li2019}, and \cite{bai2019}

To solve the problem we propose\footnote{\cite{riba2018} already used an implicit matching for predicting the GED with neural networks but we will improve on their work by utilizing a classical cost matrix and modern matching tools.} to match the nodes of both graphs, similar to classical GED algorithms (\citealp{hungarian2009}; \citealp{frankhauser2011}). These algorithms are based on bipartite graph matching. They set up a cost Matrix $C \in \mathbb{R}^{N \times N}$, where $N = N_1 + N_2 = \vert V_1 \vert + \vert V_2 \vert $ and solve the following constrained optimization problem.

\vspace{.2cm}

\noindent
\begin{minipage}{.5\linewidth}

     \[
          C=
               \left[
               \begin{array}{ccc|ccc}
                    C_{1,1} & \dotsi & C_{1, N_2} & C_{1, \epsilon} & \dotsi & \infty \\
                    \vdots & \ddots & \vdots & \vdots & \ddots & \vdots \\
                    C_{N_1, 1} & \dotsi & C_{N_1, N_2} & \infty & \dotsi & C_{N_1, \epsilon} \\
                    \hline
                    C_{\epsilon, 1} & \dotsi & \infty & 0 & \dotsi & 0 \\
                    \vdots & \ddots & \vdots & \vdots & \ddots & \vdots \\
                    \infty & \dotsi & C_{\epsilon, N_2} & 0 & \dotsi & 0 \\
               \end{array}
               \right]
     \]

\end{minipage}%
\begin{minipage}{.5\linewidth}

     \begin{equation}
          \begin{gathered}
               % \min \sum_{i = 1}^{N} \sum_{j = 1}^{N} M_{ij} C_{ij} = d(M, C) \\
               d(C) = \min \langle M, C \rangle_\mathrm{F} \\
               \text{subject to} \\
               \sum_{i = 1}^{N} M_{ij} = 1 \forall j \in \{1 \dots N\} \\
               \sum_{j = 1}^{N} M_{ij} = 1 \forall i \in \{1 \dots N\} \\
               M_{ij} \in \{0, 1\}
          \end{gathered}
     \end{equation}

\end{minipage}


\vspace{.2cm}

Here $\langle \cdot, \cdot \rangle_\mathrm{F}$ denotes the frobenius inner product. $C_{i, j}$ is the cost for replacing node $i$ with node $j$, $C_{\epsilon, j}$ the cost for inserting node $j$, and $C_{i, \epsilon}$ the cost for deleting node $i$. These costs depend on the direct neighborhood of the node, and can even be extend with random walks around it, but they still lack accuracy on graphs with complex global structures (\citealp{hungarian2009}). The assignment problem can be solved exactly by algorithms such as the Hungarian method, also called the Kuhn-Munkres algorithm. (\citealp{hungarian1955}) or the VJ algorithm (\citealp{vj1987}).

In order to allow the model to learn arbitrary metrics on graphs we use a message passing network to learn the entries of the cost matrix $C$. The network generates node embeddings that contain information about their local neighborhood. Instead of aggregating this information we compute pairwise distances between nodes of both graphs yielding a trainable equivalent to the upper left bock of the cost matrix. The corresponding values for insertion and deletion can be computed by taking a norm of the embeddings. Analog to the classical approach we then compute the distance as $ d = \langle M, C \rangle_\mathrm{F} $.

For finding a matching there are multiple possibilities:
\begin{itemize}
     \itemsep0em
     \item \textbf{Nearest Neighbor.} Taking the sum of the row-wise minima (and, for the sake of symmetry, the sum of the column-wise minima) yields what is known in computer vision as the \textbf{Chamfer Distance}. The matching runs in $O(N^2)$ and can be implemented using the $\argmax$ operator which can be backpropagated and is already implemented in modern deep learning frameworks.
     \item \textbf{Optimal Assignment.} Applying the Kuhn-Munkres algorithm results in an optimal assignment, $M_{ij} \in \{0,1\}$, in roughly $O(N^3)$ operations. However, this is a discrete algorithm, and therefore not differentiable.
     \item \textbf{Optimal Transport.} The optimal transport problem is a continuos relaxation of the optimal assignment problem, where $M_{ij} \in \left[ 0,1 \right]$. In this setting usually called the \textbf{Earth Mover Distance} it can be found in roughly $O(N^3 \log(N))$ operations.
     \item \textbf{Regularized Optimal Transport.} \cite{sinkhorn2013} introduced an algorithm for solving an entropy regularized version of the optimal transport problem minimizing $\langle M, C \rangle_\mathrm{F} + \frac{1}{\lambda} \mathrm{H}(M)$. One simply computes the kernel matrix $ K = e^{ -\frac{C}{\lambda}}$ and applies Sinkhorn’s iterative matrix scaling. This is called the \textbf{Sinkhorn Distance} and can be computed extremely fast in practice with worst case running time of $O(N^2)$.
\end{itemize}

While the chamfer distance can easily be backpropagated, we would prefer not to backpropagate through the iterative sinkhorn scaling or the cubic earth mover distance. It turns out that we can apply Danskin's theorem (\citealp{danskin1967}) to calculate analytic gradients for the optimization problems above.
\begin{theorem}
     If $\phi(x,z)$ is a continuous function of two arguments, $\phi: {\mathbb R}^n \times Z \rightarrow {\mathbb R}$ where $Z \subset {\mathbb R}^m$ is a compact set, and  $\phi(x,z)$ is convex in $x$ for every $z \in Z$.
     Then the derivate of the function $f(x) = \max_{z \in Z} \phi(x,z)$ is:
     \begin{equation}
          D f(x) = \phi'(x,z^*)  \text{, where } z^*(x) = \argmax_{z \in Z}(\phi(x,z))
     \end{equation}
\end{theorem}
Since $d(C, M)$ is affine, $-d(M,C)$ is also affine and convex. Moreover, the set of all possible matchings $M$ is compact. Then, we can apply Danskin's theorem which yields that the derivate of $d(C) = \min \langle M, C \rangle_\mathrm{F}$ is the derivate of the frobenius inner product between $C$ and the optimal matching $M$, which we have computed already for the forward pass. And the final gradient is simply the matrix $M$ itself. Armed with the analytic gradient we can backpropagate any of the distances mentioned above, and learn in end-to-end fashion.

Preliminary experiments indicated that the sinkhorn distance is not only similar fast to compute as the chamfer distance, but also yields better results than the other much slower methods. The regularization results into softer matchings compared to hard assignments, or the still very peaky matchings of the earth mover's distance. We believe these soft matchings propagate more gradient information to the message passing network allowing it to thoroughly adapt to the given distribution of graphs. Note that by adjusting the regularization parameter $\lambda$ we can control how soft the matchings are.

The final objective function will then depend on the specific task. For learning the GED we simply use the mean squared error between prediction and ground truth values. More specialized objectives such as contrastive loss proposed by \cite{riba2018} for keyword spotting, or pairwise margin loss, or triplet loss proposed by \cite{li2019} for detecting fraudulent binaries can also be used with our model.

In response to \cite{fey2020_update}, who used a very similar model for graph matching, we decided to evaluate our model on a graph matching task as well. In this case, where the matching matrix is used directly in the objective function we are not able to apply Danskin's theorem. Therefore, we do have to backpropagate through the sinkhorn iterations. However, we found that using 50 iterations is sufficient to get a good matching and can still be backpropagated reasonably fast.

% Note wikipedia gives best running times as $O(mn + n^2 * \log(n))$ for Munkres
% and $O(N \log(N)^2)$ for EMD


\section{The Graph Distance Network}

We propose the Graph Distance Network (GDN), which uses a siamese network structure with multiple shared message passing layers, the previously derived learnable cost matrix, and the Sinkhorn distance with analytic gradients.

Let $d_h$ be an arbitrary embedding size, $d_{\text{node}}$ the number of node features of the input graphs, $d_{\text{edge}}$ the number of edge types, $W_l \in \mathbb{R}^{d_h \times d_h}$, $b_l \in   \mathbb{R}^{d_h}$ a weight matrix and bias, $h_i^l \in \mathbb{R}^{d_h}$ the embedding of node $i$ at layer $l$ for $l > 0$, $\sigma(\cdot)$ an elementwise non-linearity, and $E \in \mathbb{R}^{d_{\text{edge}} \times d_h \times d_h}$ a weight tensor ($h_i^0 = v_i \in \mathbb{R}^{d_{\text{node}}}$, $W_0 \in \mathbb{R}^{d_{\text{node}} \times d_h}$, $b_0 \in \mathbb{R}^{d_{\text{node}}}$). We then use the following message passing function:
\begin{equation}
     h_i^{l} = \sigma(W_{l} h_i^{l-1} + b_l) + \sum_{j \rightarrow i} \sigma(W_{l} h_j^{l-1} + b_l) E_{e_{j \rightarrow i}}
\end{equation}
where $E_{e_{j \rightarrow i}} \in \mathbb{R}^{d_h \times d_h}$ is the weight tensor indexed by the discrete feature of the edge connecting node $j$ with node $i$. As activation function $\sigma(\cdot)$ we use the $\operatorname{leaky-relu}$ non-linearity.

To compute the entries of the cost matrix we concatenate all node embeddings and take the norm of pairwise distances across both graphs, or the embeddings scaled by a trainable $\alpha \in \mathbb{R}^{d_\text{final}}$:
\begin{equation}
     \begin{gathered}
          h_i^\text{final} = [h_i^L ; \dots ; h_i^1 ; \sigma(W_{1} h_i^{0} + b_0)] \in \mathbb{R}^{d_\text{final}}\\
          C_{i,j} = \norm{h_i^\text{final} - h_j^\text{final}} \quad
          C_{i, \epsilon} = \norm{ \alpha h_i^\text{final}} \quad
          C_{\epsilon, j} = \norm{ \alpha h_j^\text{final}} \quad i \in \{1 \dots N_1\}, j \in \{1 \dots N_2\}
     \end{gathered}
\end{equation}

In the first experiment we tested three different norms: the euclidean norm ($p=2$), the manhattan norm ($p=1$), and a simple two layer perceptron (MLP). The first layer of this MLP does not change the embedding size, then we apply a $\operatorname{relu}$ non-linearity. The second layer reduces the embedding size to a single number to which we apply the $\operatorname{softplus}$ non-linearity, $\ln(1 + e^x)$.

As for the matrix $C$ itself we tested two variants. The first one we call BP-Distance on account of bipartite graph matching. Due to implementation advantages we decided to pad the pairwise distances with rows and columns of $C_{i, \epsilon}, C_{\epsilon, j}$ respectively, instead of the $\infty$-values typically used.
\begin{equation}
     C_\text{BP}=
          \left[
          \begin{array}{ccc|ccc}
               C_{1,1} & \dotsi & C_{1, N_2} & C_{1, \epsilon} & \dotsi & C_{1, \epsilon} \\
               \vdots & \ddots & \vdots & \vdots & \ddots & \vdots \\
               C_{N_1, 1} & \dotsi & C_{N_1, N_2} & C_{N_1, \epsilon} & \dotsi & C_{N_1, \epsilon} \\
               \hline
               C_{\epsilon, 1} & \dotsi & C_{\epsilon, N_2} & 0 & \dotsi & 0 \\
               \vdots & \ddots & \vdots & \vdots & \ddots & \vdots \\
               C_{\epsilon, 1} & \dotsi & C_{\epsilon, N_2} & 0 & \dotsi & 0 \\
          \end{array}
          \right]
     \in \mathbb{R}^{(N_1 + N_2) \times (N_1 + N_2)}
\end{equation}
This does lead to different results after Sinkhorn normalization due to the entropy regularization, but the network can easily adapt by increasing $\norm{\alpha}$. Therefore, this modification should not affect our trainable model. The second variant of the cost matrix is the smallest square sub-matrix of the one above that contains all pairwise distances, such that \mbox{$C_\text{no-BP} \in \mathbb{R}^{\max({N_1, N_2}) \times \max({N_1, N_2})}$}.

The final distance $d$ is then computed and trained via the mean squared error between target distance and prediction:
\begin{equation}
     \begin{gathered}
          M = \operatorname{sinkhorn}\left(e^{-\frac{C}{\lambda}}\right) \\
          d = \langle M, C \rangle_\mathrm{F} \\
          \text{loss} = (d - d_\text{target})^2
     \end{gathered}
\end{equation}

Implementation. Our GNN is implemented with PyTorch (\citealp{pytorch}), PyTorch Scatter, and PyTorch-geometric (\citealp{pytorchgeometric}) and can process sparse mini-batches with parallel GPU acceleration. We have two implementations for the matching. The first version pads matrices to form batches and can be used in cases where all graphs have a similar number of nodes. The second version uses sparse matrices in COOrdinate format, which is less efficient with evenly sized graphs but can be crucial for parallelizing workloads with skewed graph size distributions.


\section{Experiment Setup}

% Dataset
For our first experiment we use synthetic preferential attachment graphs (\citealp{pref_att2002}) and real-world chemical molecules of the AIDS dataset\footnote{\url{https://wiki.nci.nih.gov/display/NCIDTPdata/AIDS+Antiviral+Screen+Data}}. Both datasets have discrete node and edge features, and graphs are limited to at most 30 nodes. We provide some basic dataset statists in the appendix (Table \ref{tab:ex1-data}).

% Baseline Setup
We compare our results to the implementations of \cite{riba2018}, \cite{bai2019}, and \cite{li2019} in Table \ref{tab:ex1-baselines}. All methods are described in the Related Work Section. We train for 500 epochs and report the best error on the validation set and the corresponding test error. We ran 3 trials of every setting and searched over the cross product of 2 different batch sizes (depending on the implementation), 32 and 64-dimensional node embeddings, 1 and 3 layers of message passing, 4 learning rates, and 4 degrees of regularization. Implementation details can be found in the respective GitLab repositories\footnote{Riba: \url{https://gitlab.lrz.de/ge98beq/siamese_ged}, Bai \url{https://gitlab.lrz.de/ge98beq/simgnn}, Li \url{https://gitlab.lrz.de/ge98beq/gmn/tree/gr_report}}.
% Adam/SGD

% GDN Setup
For our model, the graph distance network (GDN), we fixed the batch size to 1024, the embedding size to 32, and the number of layers to 3, and added 3 choices for the Sinkhorn regularization constant $\lambda$. Moreover, we include an ablation study over the choice of the cost matrix and different norms in Table \ref{tab:ex1-ablation}.


\section{Results}



% Discussion
Table \ref{tab:ex1-baselines} shows that our method clearly outperforms all baselines. Best results on both datasets were found with the no-BP cost matrix, MLP-norm, and small weight decay for regularization. Interestingly, the only difference in the configurations is the  Sinkhorn regularization constant $\lambda$, which is smaller for the synthetic graphs. However, somewhat concerning is the gap between the validation and test performance on the real-world dataset, which also appears with the model of \cite{bai2019}. This phenomenon will be discussed in the ablation study.


\begin{table}[htbp]
    \addtolength{\tabcolsep}{-1pt}
    \fontsize{9pt}{10.25pt}\selectfont
    \centering
    \renewcommand{\arraystretch}{1.2}
    \begin{tabular}{|l|c|c|c|c|}
        \hline
        \multirow{2}{*}{} & \multicolumn{2}{c|}{Pref-Attachment} & \multicolumn{2}{c|}{AIDS} \\ \hhline{|~|-|-|-|-|}
        & Val & Test & Val & Test \\ \hhline{|=|=|=|=|=|}
        \cite{riba2018} & $12.2 \pm 0.2$ & $12.1 \pm 0.6$ & $15.5 \pm 0.3$  & $15.6 \pm 0.3$ \\ \hline
        \cite{bai2019} & $7.7 \pm 1.0$ & $9.6 \pm 2.5$ & $4.2 \pm 0.3$ & $8.7 \pm 0.1$ \\ \hline
        \cite{li2019} & $5.5 \pm 0.1 $ & $7.8 \pm 0.3$ & $10.6 \pm 0.3$ & $11.7 \pm 0.9$ \\ \hline
        GDN & $4.2 \pm 0.1$ & $\boldsymbol{4.5 \pm 0.2}$ & $3.3 \pm 0.04$ & $\boldsymbol{6.2 \pm 1.0}$ \\ \hline
    \end{tabular}
    \caption{RMSE to ground truth GED with standard deviation across 3 runs.}
    \label{tab:ex1-baselines}
\end{table}



The ablation study in Table \ref{tab:ex1-ablation} indicates that:
\begin{itemize}
    \itemsep0em
    \item \textbf{The Euclidean norm is superior to the manhattan norm, and a trainable MLP can improve results even further.}
    \item \textbf{There comes no immediate benefit from the full-size cost matrix.} In all but one case the no-BP cost matrix is at least as good as the BP variant in validation and test performance. Even though intuitively it feels convenient to have the option to explicitly add or remove nodes, it may be the case that we can get the same result by substituting one node for another in the case of the GED. However, using the no-BP matrix could certainly lead to problems when attempting to learn "sparse matchings" where only a few nodes have a match in the other graph. Moreover, it is not surprising that the results of both variants are very similar because the Sinkhorn normalization will place a lot of weight onto the 0 entries in the cost matrix. It does that to a point where summing over the part of the rows (or columns depending on which graph has more nodes) which had 0 entries before normalization yields values close to 1. Therefore, the values that contribute to the distance come mostly from entries which also show up in our no-BP variant matrix.
    \item \textbf{The AIDS validation set does not properly represent the entire distribution.} The difference in validation and test performance on the synthetic graphs can be explained by the variation between consecutive epochs. Since we stop on the best validation result it is no surprise that the test result at that exact point is slightly worse on average. In contrast, the massive gap in performance, and the high standard deviation on the AIDS test set indicate a shift of distributions between the validation and the test set. Some runs generalized well from the training data to the validation set, but they consistently performed worse on the test set across all epochs. Interestingly, not all runs behaved that way. We also collected many runs where the test performance is close to validation performance\footnote{Often these runs had higher Sinkhorn regularization but we did not run further experiments to verify that impression.}. Apparently the manhattan norm is more susceptible to overfit in this manner than the MLP norm.
\end{itemize}


\begin{table}[htbp]
    \addtolength{\tabcolsep}{-1pt}
    \fontsize{9pt}{10.25pt}\selectfont
    \centering
    \renewcommand{\arraystretch}{1.2}
    \begin{tabular}{|c|c|c|c|c|c|}
        \hline
        \multicolumn{2}{|c|}{} & \multicolumn{2}{c|}{Pref-Attachment} & \multicolumn{2}{c|}{AIDS} \\ \hline
        BP & Norm & Val & Test & Val & Test \\ \hhline{|=|=|=|=|=|=|}
        \multirow{3}{*}{yes} & $p=1$ & $5.2 \pm 0.1$ & $6.4 \pm 0.7$ & $3.9 \pm 0.1$ & $17.1 \pm 10.9$ \\ \hhline{|~|-|-|-|-|-|}
        & $p=2$ & $4.8 \pm 0.1$ & $5.7 \pm 0.2$ & $3.9 \pm 0.2$ & $8.6 \pm 2.6$ \\ \hhline{|~|-|-|-|-|-|}
        & MLP & $4.5 \pm 0.1$ & $\boldsymbol{4.5 \pm 0.3}$ & $3.5 \pm 0.03$ & $\boldsymbol{4.7 \pm 0.1}$ \\ \hline
        \multirow{3}{*}{no}  & $p=1$ & $4.9 \pm 0.2$ & $5.8 \pm 0.2$ & $3.9 \pm 0.3$ & $16.2 \pm 6.0$ \\ \hhline{|~|-|-|-|-|-|}
        & $p=2$ & $4.8 \pm 0.1$ & $5.7 \pm 0.6$ & $3.7 \pm 0.2$ & $6.8 \pm 3.4$ \\ \hhline{|~|-|-|-|-|-|}
        & MLP & $4.2 \pm 0.1$ & $\boldsymbol{4.5 \pm 0.2}$ & $3.3 \pm 0.04$ & $6.2 \pm 1.0$ \\ \hline
    \end{tabular}
    \caption{Ablation Study: RMSE with standard deviation across 3 runs.}
    \label{tab:ex1-ablation}
\end{table}


% Note bai has those 115 / 54 mse's

% Best settings?
% 0.0005 weight decay (not 0!)
% big learning rate
% sinkhorn_reg=0.2



\section{Graph Matching}

As described in \ref{section:related_work} Related Work, \cite{fey2020_update} have recently proposed a similar model, the Deep Graph Consensus (DGC), to predict soft matchings between nodes. They train on the negative log-likelihood between the soft matching and some ground truth matching (following \citealp{wang2019}). The final performance is measured by how often the row-wise maximum in the soft matching yields the correct match (Hits@1).

Wanting to show that it is advantageous to apply the Sinkhorn normalization to the kernel matrix instead of the cost matrix directly, we decided to also run the keypoint matching experiment of \cite{fey2020_update} on the PASCALVOC  dataset \cite{pascal2010} with Berkeley annotations \cite{annotations2009}.

To align our model as much as possible with their work we use the same SplineCNN (\citealp{spline2018}) with the same embedding size and number of layers to generate node embeddings. We use $p=2$ for our norm (to avoid using additional parameters), and the no-BP variant of the cost matrix because the graphs are built such that every node in the first (source) graph has a match in the second (target) graph. Therefore, the network would only learn to predict 0’s for the additional values. This left us with only the learning rate and the regularization constant $\lambda$ to tune. For numeric stabilization we scaled the kernel matrix by the sum of all entries divided by the number of entries: $\text{reg}^l = \frac{\norm{C^l}_1}{\max(N_1,N_2)^2}$. We also implemented a consensus step. For a detailed comparison of the two models see \mbox{Appendix \ref{appendix:consensus}}.

In Table \ref{tab:ex2} we report results on the test set of the model with the best validation performance across 18 runs (3 trials, 2 learning rates, and 3 regularization constants), and compare these to the numbers reported by \cite{fey2020_update}. The table shows that using the kernel matrix and the inherent ability to tune the regularization constant improves the base model without consensus. The 3 percent points gain constitute over half of the gains made by the original consensus step. Unfortunately, applying the consensus step to our model was not successful. In fact, by adding our consensus steps the performance drops slightly. We believe that the consensus step should be orthogonal to the gains due to the kernel matrix. However, so far, we were not able to improve upon the poor performance even after multiple tweaks to the cost matrix generation.

\begin{table}[ht]
    \centering
    \resizebox{\textwidth}{!}{\begin{tabular}{ccccccccccccccccccccccc}
        \hline
        \textbf{Method}& &\textbf{Aero}&\textbf{Bike}&\textbf{Bird}&\textbf{Boat}&\textbf{Bottle}&\textbf{Bus}&\textbf{Car}&\textbf{Cat}&\textbf{Chair}&\textbf{Cow}&\textbf{Table}&\textbf{Dog}&\textbf{Horse}&\textbf{M-Bike}&\textbf{Person}&\textbf{Plant}&\textbf{Sheep}&\textbf{Sofa}&\textbf{Train}&\textbf{TV}&\textbf{Mean} \\
        \hline
        DGC & L=0 &42.1&57.5&49.6&59.4&83.8&84.0&78.4&67.5&37.3&60.4&85.0&58.0&66.0&54.1&52.6&93.9&60.2&85.6&87.8&82.5&67.3\\
        \hline
        GDN & L=0&45.3&63.8&55.5&72.6&88.9&88.5&79.5&69.1&36.8&58.5&82.8&59.2&70.1&59.8&53.4&96.3&63.7&92.5&91.7&85.0&70.5\\
        \hline
        DGC & L=10&45.5&67.6&56.5&66.8&86.9&85.2&84.2&73.0&43.6&66.0&92.3&64.0&79.8&56.6&56.1&95.4&64.4&95.0&91.3&86.3&72.8\\
        \hline
        GDN & L=10&40.2&59.6&50.1&65.0&87.8&86.7&82.3&69.6&37.4&59.0&93.1&56.7&73.1&52.9&58.5&96.3&62.1&90.0&93.4&87.3&70.0\\
        \hline
    \end{tabular}}
    \caption{Hits@1 (\%) on the PASCALVOC dataset with Berkeley keypoint annotations}
    \label{tab:ex2}
\end{table}

% 0.4016 0.5957 0.5005 0.6497 0.8784 0.8669 0.8227 0.6962 0.3740 0.5895 0.9310 0.5673 0.7313 0.5288 0.5853 0.9634 0.6207 0.8955 0.9337 0.8725
% 0.7003

% "config" : {
%     "alpha_vector" : true,
%     "batch_size" : 512,
%     "bp_dist_matrix" : false,
%     "consensus_type" : "none",
%     "dataname" : "pascal",
%     "db_collection" : "gdn_match_report_test2",
%     "device" : "cuda",
%     "emb_dist_p" : 2,
%     "gnn_type" : "spline",
%     "graph_distance" : "matching",
%     "learning_rate" : 0.01,
%     "max_epochs" : 100,
%     "model_type" : "gdn",
%     "overwrite" : 45,
%     "patience" : 5,
%     "run" : 0,
%     "seed" : 318116669,
%     "sinkhorn_reg" : 0.1,
%     "sparse_batching" : false
% },
% Epoch 13/99,    train    loss: 11.3083, hits@1: 0.7304 (5.90s)
% 2020-03-04 12:50:46 (INFO): Epoch 13/99,    val      0.4341 0.5750 0.6551 0.6753 0.8093 0.7804 0.7854 0.6490 0.4232 0.6160 0.8913 0.6165 0.6332 0.6561 0.6108 0.9710 0.6804 0.6545 0.8833 0.8951
% 2020-03-04 12:50:46 (INFO): Epoch 13/99,    val      loss: 13.9017, hits@1: 0.6948 (3.39s)
% 2020-03-04 12:50:49 (INFO): Epoch 13/99,    test     0.4533 0.6383 0.5547 0.7259 0.8890 0.8506 0.7945 0.6909 0.3682 0.5850 0.8276 0.5924 0.7071 0.5984 0.5341 0.9634 0.6371 0.9254 0.9171 0.8500
% 2020-03-04 12:50:49 (INFO): Epoch 13/99,    test     loss: 13.2461, hits@1: 0.7052 (3.38s)

% 0.4533 0.6383 0.5547 0.7259 0.8890 0.8506 0.7945 0.6909 0.3682 0.5850 0.8276 0.5924 0.7071 0.5984 0.5341 0.9634 0.6371 0.9254 0.9171 0.8500
% 0.7052




% "config" : {
%         "alpha_vector" : true,
%         "batch_size" : 512,
%         "bp_dist_matrix" : false,
%         "consensus_emb_size" : 128,
%         "consensus_nlayers" : 10,
%         "consensus_p_norm" : 777,
%         "consensus_type" : "spline",
%         "dataname" : "pascal",
%         "db_collection" : "gdn_match_report_test2",
%         "device" : "cuda",
%         "emb_dist_p" : 2,
%         "gnn_type" : "spline",
%         "graph_distance" : "matching",
%         "learning_rate" : 0.01,
%         "max_epochs" : 100,
%         "model_type" : "gdn",
%         "overwrite" : 124,
%         "patience" : 5,
%         "run" : 0,
%         "seed" : 147824079,
%         "sinkhorn_reg" : 0.05,
%         "sparse_batching" : false
%     },
% 13/99,    train    loss: 22.2035, hits@1: 0.7305 (33.48s)
% 2020-03-04 14:03:46 (INFO): Epoch 13/99,    val      0.3795 0.5732 0.6840 0.7226 0.8725 0.8210 0.7388 0.6151 0.4173 0.5593 0.9130 0.6505 0.6246 0.5844 0.6165 0.9574 0.6769 0.5864 0.9389 0.9244
% 2020-03-04 14:03:46 (INFO): Epoch 13/99,    val      loss: 29.4006, hits@1: 0.6928 (9.32s)
% 2020-03-04 14:03:56 (INFO): Epoch 13/99,    test     0.4016 0.5957 0.5005 0.6497 0.8784 0.8669 0.8227 0.6962 0.3740 0.5895 0.9310 0.5673 0.7313 0.5288 0.5853 0.9634 0.6207 0.8955 0.9337 0.8725
% 2020-03-04 14:03:56 (INFO): Epoch 13/99,    test     loss: 27.7569, hits@1: 0.7003 (9.29s)\n

% 0.4016 0.5957 0.5005 0.6497 0.8784 0.8669 0.8227 0.6962 0.3740 0.5895 0.9310 0.5673 0.7313 0.5288 0.5853 0.9634 0.6207 0.8955 0.9337 0.8725
% 0.7003


\section{Conclusion}

% cost matrix

We proposed to use a cost matrix, motived by bipartite graph matching, and an explicit soft matching between nodes to learn an accurate and scalable estimation of metrics on graphs. Our model has the same theoretical complexity $\mathcal{O}(N_1 N_2)$ as previous neural network methods designed for this task. In practice the model runs quickly because our implementation can efficiently handle large mini-batches of arbitrarily sized graphs making it well suited for modern GPU setups. We showed empirically that the model improves upon state-of-the-art methods in predicting the ubiquitous graph edit distance.

% Using matching for distance computation has great potential. Can compute matching fast and still be fully differentiable. Works great on GED.
% no reason why we couldn't apply it to other distances


Closely following the setup of \cite{fey2020_update}, we applied our model to graph matching. In particular, motivated by the Sinkhorn distance, we proposed to apply Sinkhorn scaling to the kernel of the cost matrix. We showed empirically that with little tuning of the regularization constant this model improves upon the best results of \cite{fey2020_update} without the consensus step.

% mention that full-BP can be applied to sparse matching


\section{Outlook}

Our model should be capable of learning arbitrary metrics on graphs. While metric can be diverse, there are some requirements for a function to be considered a metric in mathemetics.
\begin{itemize}
    \itemsep0em
    \item $d(x,y) \ge 0$ non-negativity
    \item $d(x,y) = 0 \Leftrightarrow x = y$ identity of indiscernibles
    \item $d(x,y)  = d(y,x)$ symmetry
    \item $d(x,y) \le d(x,z) + d(z, y)$ subadditivity or triangle inequality
\end{itemize}
We already ensured non-negativity, and symmetry, but it would certainly be interesting to investigate if we can ensure the triangle inequality. Perhaps that could lead to a small modification which induces a strong and useful prior. Also investigating the identity of indiscernibles could lead to interesting results. If it was possible to ensure this property it would solve the graph isomorphism problem, and finally answer the question of whether the problem is NP-complete or not. Maybe easier to handle and therefore more promising is translational invariance, $d(x,y) = d(x+a,y+a)$, which might be a valid assumptions in many cases. Graph sums can be defined by adding the adjacency matricies of equal sized graphs \cite{graph_sum2004} or, in the context of the graph edit distance, maybe by adding a node that is connected to all other nodes.

A different direction for further research is to speed up the current model. It seems inevidable that the whole process is quadratic in the number of nodes due to the use of the cost matrix $C \in \mathbb{R}^{N \times N}$. However, we might actually be albe to get linear run time by using the nystr{\"{o}}m method or multiscale matching. \cite{nytrom2019} show that using the nystr{\"{o}}m method and sinkhorn scaling one can accurately approximate the sinkhorn distance without ever computing the full cost matrix. The nystrom method is based on a set of landmarks that is then used to speed up matching. Similarly mlutiscale matching uses clusters of nodes that are then matched with each other  (\citealp{}??). One problem with both approaches is that they only produce the final distance and not an explicit matching, which would be required to apply Danskin's thoerom for fast backpropagation. Therefore, the key for implementing both of these methods will be to find an efficient way to backpropagate the gradients. Ideally one would combine both approaches since landmarks and clusters each have their merits and demerits. Using landmarks one can vastly overestimate small distances, and using clusters longer distances become inaccurate due to repeated averaging.


\newpage

\bibliographystyle{plainnat}
\bibliography{egbib}

% \cite    -> Name (year)
% \citealp -> Name, year

\newpage

\appendix
\appendixpage

In this report I summarize the work I did during my guided research. The main part of the report focuses on findings that are relevant to a potential paper submission, but here in the appendix I will add some information about the datasets we used, and some we did not use (for completeness).


\section{Graph Edit Distance Datasets}
\begin{table}[htbp]
    \addtolength{\tabcolsep}{-1pt}
    \fontsize{9pt}{10.25pt}\selectfont
    \centering
    \renewcommand{\arraystretch}{1.2}
    \begin{tabular}{|l|c|c|c|c|c|c|c|}
        \hline
        Dataset & Node Feat & Edge Feat & \#Train & \#Val/Test & Max Nodes & Avg Nodes & Avg Edges \\
        \hline
        Pref-Att & 6 & 4 & 144 & 48 & 30 & 20.6 & 75.4 \\ % edges mean 75.4
        \hline
        AIDS & 53 & 4 & 144 & 48 & 30 & 20.6 & 44.6 \\ % edges mean 44.6
        \hline
    \end{tabular}
    \caption{Datasets}
    \label{tab:ex1-data}
\end{table}

Note \#Train and \#Val/Test are the number of graphs. We train and validate/test on all possible pairs of graphs, except the pairs where a graph is paired with itself.

\section{Yeast Dataset}

The Yeast Dataset (\url{https://www3.nd.edu/~cone/MAGNA++/}), used by \cite{yeast2019}, contains the protein-protein interaction (PPI) network of yeast and noise versions thereof. The base network contains 1,004 proteins and 4,920 high-confidence interactions. The noisy versions contain additional low-confidence interactions.

The first obstacle with this dataset would have been the generation of train/val/test splits.  Moreover, we could not have compared our results directly with \cite{yeast2019} because they learn unsupervised. The second obstacle is that the dataset contains no node or edge features. That is mostly a problem because GNNs are particularly good at using node, and also edge features. Therefore, it is questionable if a  model based on GNNs would perform impressive enough. In the end we decided not to use this dataset, also because we discovered the paper by \cite{fey2020_update} who use plenty of graph matching datasets that fit our model much better.

\section{Control Flow Dataset}\

\cite{li2019} used control flow graphs to evaluate their model. They used metric learning (margin-loss with pairs, and triplets) to recognize a function compiled from different compliers as similar, and different functions as not similar. They didn't share their dataset of FFmpeg functions with us, but referred us to a github repository, which I then forked (\url{https://github.com/johannespitz/functionsimsearch/tree/master}) and used to generate the UnRAR datasets, which (at the time of writing) can be found at '/nfs/students/pitz/cfgs' on the file server. To generate the graphcollections I used the notebook 'cfgs.ipynb' in the graph-distance repository. Note: use the .pickle version because it is much faster than .npz.

Currently we have a few different options for train/val/test splits. The simplest split is to use train\_all/val/test, which is an 80/10/10 split in terms of functions (not graphs). The attraction list contains all pairs of graphs that are from the same function, and the repulsion list contains as many negative samples. The train\_all split is then further split in: Option1, train\_across/val\_across/test\_acorss. Here out of the n graphs of any function we pick one to be the test graph, and all pairs between it and the remaining n-1 graphs are placed in the attraction list. Of the n-1 we then draw a validation graph and again all pairs between it and the remaining n-2 graphs are placed in the attraction list. The attraction list of the train\_across split contains then all possible pairs of the remaining n-2 graphs. Option2, train12/val1/val2/test1/test2. Here we split train\_all as chunk/val2/test2 (70/15/15) in terms of graphs (not functions), while ensuring that at least one graph of each function is contained the chunk. We then generate all positive pairs, but for the chunk we split those again into train12/val1/test1 (80/10/10) ensuring that every graph is in train12 at least once. All repulsion list for both options contain as many negative samples as the corresponding attraction list. Triplets are only available with the simple  train\_all/val/test (80/10/10) split.

Our model can be trained with pairs or triplets as it is. I did not include any results with the dataset in this report because the baselines cannot be trained on it (without putting in additional work).

\section{Consensus Step}
\label{appendix:consensus}

The main difference between our model and \cite{fey2020_update} is that they normalize the cost matrix $C$ instead of kernel matrix $e^{-\frac{C}{\lambda}}$. Consequently, they report bad gradients with Sinkhorn normalization and use row-wise sofmax instead. For their cost matrix they only use the pairwise distances between nodes without padding with norms, which should not be a problem for tasks where each node in the first graph has a match in the second graph, but could be problematic in other settings. In Table \ref{tab:diff-models} we collected a complete list of the differences between both models:


\begin{table}[htbp]
    \addtolength{\tabcolsep}{-1pt}
    \fontsize{9pt}{10.25pt}\selectfont
    \centering
    \renewcommand{\arraystretch}{1.2}
    \begin{tabular}{|C{1cm}|c|c|}
        \hline
        & DGC & GDN \\
        \hline
        $C^0_{i,j} $ & $h_i^0 \cdot h_j^0$ & $\begin{cases}
            \norm{h_i^0 - h_j^0}_2 & \text{if}\ 1 \leq i \leq N_1,\ 1 \leq j \leq N_2\\
            \norm{\alpha h_i^0}_2 & \text{if}\ N_1 < i \leq N_2,\ 1 \leq j \leq N_2
         \end{cases}$ \\
        % \hline
        $C^l_{i,j} $ & $C^{l - 1}_{i,j} + \text{MLP}(h_i^l - h_j^l)$ & $
        \begin{cases}
            \beta^l C^0_{i,j} + (1 - \beta^l) \text{MLP}(h_i^l - h_j^l) & \text{if}\ 1 \leq i \leq N_1,\ 1 \leq j \leq N_2\\
            \gamma^l \gamma^l C^{l - 1}_{i,j}  & \text{if}\ N_1 < i \leq N_2,\ 1 \leq j \leq N_2
        \end{cases}
        $ \\
        % \hline
        $M^l $ & $\operatorname{softmax}(C^l)$ & $\operatorname{sinkhorn}\left(e^{-\frac{C^l * \text{reg}^l}{\lambda}}\right)$ \\
        \hline
    \end{tabular}
    \caption{Differences of the two models}
    \label{tab:diff-models}
\end{table}

Note1: All superscripts indicate the layer of the consensus step.
Note2: $\gamma^l \in \mathbb{R}$ is a trainable parameter initialized with 1.
Note3: \cite{fey2020_update} use only those pairs where every node in the source graph finds a match in the target graph (therefore: $N_1 < N_2$). And since we use the no-BP cost matrix we have $1 \leq i, j \leq N_2$

% $$ \begin{cases}
%     \beta^l C^0_{i,j} + (1 - \beta^l) \text{MLP}(h_i - h_j) & \text{if}\ 1 \leq i \leq N_1,\ 1 \leq j \leq N_2\\
%     \gamma^2 C^0_{i,j}  & \text{if}\ i > N_1,\ 1 \leq j \leq N_2\\
%     \gamma^2 C^0_{i,j} & \text{if}\ 1 \leq i \leq N_1,\ j > N_2
%  \end{cases}
%  $$

% \begin{alignat*}{3}
%      & && \quad \text{\cite{fey2020_update}} && \quad \text{Ours} \\
%      &C^0_{i,j} &&= h_i \cdot h_j &&= \norm{h_i - h_j}_2 \qquad \text{(fill with norms)} \\
%      &C^l_{i,j} &&= C^{l - 1}_{i,j} + \text{MLP}(h_i - h_j) &&= \beta_l C^0_{i,j} + (1 - \beta_l) \text{MLP}(h_i - h_j) \qquad \text{(keep old norms scaled by } \alpha^2 \text{)} \\
%      &M^l &&= \operatorname{softmax}(C^l) &&= \operatorname{sinkhorn}(e^{-\frac{C^l * \text{reg}}{\lambda}}) \\
% \end{alignat*}

% We also implemented a consensus step based on their work. We use instead of
% \\
% $C_0 = h_s \cdot h_t$\\
% $C_0 = \norm{h_s - h_t}_2$ fill with norms\\
% $C_l = C_{l - 1} + \text{MLP}(h_s - h_t)$\\
% $C_l = \beta_l C_0 + (1 - \beta_l) \text{MLP}(h_s - h_t)$ keep old norms scaled by $\alpha^2$\\
% $M_l = \operatorname{softmax}(C_l)$ \\
% $M_l = \operatorname{sinkhorn}(e^{-\frac{C_l * \text{reg}}{\lambda}}) \quad \text{reg} = \frac{\norm{C_l}_1}{\max(n1,n2)^2}$\\



%%%%%%%%%%%%%%%%%%%%%%%%%%%%%%%%%%%%%%%%%%%%%%%%%%%%%%%%%%%%%%%%%%%%%%%%%%%%%%%%%%%%%%%%%%%
%%%%%%%%%%%%%%%%%%%%%%%%%%%%%%%%%%%%%%%%%%%%%%%%%%%%%%%%%%%%%%%%%%%%%%%%%%%%%%%%%%%%%%%%%%%
%%%%%%%%%%%%%%%%%%%%%%%%%%%%%%%%%%%%%%%%%%%%%%%%%%%%%%%%%%%%%%%%%%%%%%%%%%%%%%%%%%%%%%%%%%%


% \begin{equation}
%     \begin{gathered}
%          h_i^\text{final} = [h_i^L ; \dots ; h_i^1] \in \mathbb{R}^{d_\text{final}}\\
%          C_{i,j} = \norm{h_i^\text{final} - h_j^\text{final}} \\
%          C_{i, \epsilon} = \norm{ \alpha h_i^\text{final}} \quad
%          C_{\epsilon, j} = \norm{ \alpha h_j^\text{final}} \\ i \in \{1 \dots N_1\}, j \in \{1 \dots N_2\}
%     \end{gathered}
% \end{equation}


% \begin{equation}
% C_\text{no-BP} \in \mathbb{R}^{\max({N_1, N_2}) \times \max({N_1, N_2})}
% \end{equation}

% \begin{equation}
%     \frac{\partial d}{\partial c_{i,j}} = m^*_{i,j}
% \end{equation}


% \begin{equation}
%     d(C) = \min_{M} \langle M, C \rangle_\mathrm{F} + \lambda \text{H}(M) \quad \text{s.t. } \sum_i m_{i,j} = \sum_j m_{i,j} = 1
% \end{equation}

% \begin{equation}
%     M^* = \text{Sinkhorn}(e^{-\frac{C}{\lambda}})
% \end{equation}




% \begin{table}[htbp]
%     \addtolength{\tabcolsep}{-1pt}
%     \fontsize{9pt}{10.25pt}\selectfont
%     \centering
%     \renewcommand{\arraystretch}{1.2}
%     \begin{tabular}{|l|c|c|c|c|}
%         \hline
%         \multirow{2}{*}{} & \multicolumn{2}{c|}{Pref-Attachment} & \multicolumn{2}{c|}{AIDS} \\ \hhline{|~|-|-|-|-|}
%         & Val & Test & Val & Test \\ \hhline{|=|=|=|=|=|}
%         Riba et al. (2018) & $12.2 \pm 0.2$ & $12.1 \pm 0.6$ & $15.5 \pm 0.3$  & $15.6 \pm 0.3$ \\ \hline
%         Bai et al. (2019) & $7.7 \pm 1.0$ & $9.6 \pm 2.5$ & $4.2 \pm 0.3$ & $8.7 \pm 0.1$ \\ \hline
%         Li et al. (2019) & $5.5 \pm 0.1 $ & $7.8 \pm 0.3$ & $10.6 \pm 0.3$ & $11.7 \pm 0.9$ \\ \hline
%         GDN & $4.2 \pm 0.1$ & $\boldsymbol{4.5 \pm 0.2}$ & $3.3 \pm 0.04$ & $\boldsymbol{6.2 \pm 1.0}$ \\ \hline
%     \end{tabular}
%     \caption{RMSE to ground truth GED with standard deviation across 3 runs.}
%     \label{tab:ex1-baselines}
% \end{table}


% \begin{table}[ht]
%     \centering
%    \begin{tabular}{|c|c|c|}
%         \hline
%         \textbf{Method}&\textbf{Consensus} &\textbf{Mean} \\
%         \hline
%         DGC & L=0 &67.3\\
%         \hline
%         GDN & L=0&70.5\\
%         \hline
%         DGC & L=10&72.8\\
%         \hline
%         GDN & L=10&70.0\\
%         \hline
%     \end{tabular}
%     \caption{Hits@1 (\%) on PASCALVOC}
%     \label{tab:ex2}
% \end{table}



\end{document}
