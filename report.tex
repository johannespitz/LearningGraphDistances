\documentclass[a4paper,10pt]{article}
\usepackage[utf8]{inputenc}
\usepackage[margin=3cm]{geometry}
\usepackage{amsmath}
\usepackage{amssymb}
\usepackage{amsthm}
\usepackage{fancyhdr}
\usepackage{seminar}
\usepackage{graphicx}
% \usepackage{subfigure}
\usepackage{float}
\usepackage{hyperref}

\usepackage[round]{natbib}   % omit 'round' option if you prefer square brackets

\usepackage{hhline}
\usepackage{multirow}

\usepackage{mathrsfs}
\usepackage{commath}
\usepackage{mathtools}
\usepackage{subcaption}
\usepackage{appendix}

\DeclareMathOperator*{\argmax}{arg\,max}
\DeclareMathOperator*{\argmin}{arg\,min}

\pagestyle{fancy}


%You can add theorem-like environments (e.g. remark, definition, ...) if you want
\newtheorem{theorem}{Theorem}

\title{Learning graph distances} % Replace with your title
% \newcommand{\shorttitle}{\title}
% \shorttitle{Learning graph distances}
\author{Johannes Pitz} % Replace with your name
\institute{\textit{Guided Research: Data Analytics and Machine Learning Group  \protect\\ TUM Department of Informatics}}

\makeatletter
\let\runauthor\@author
\let\runtitle\@title
\makeatother
\lhead{\runauthor}
\rhead{\runtitle}


\begin{document}

\title{Learning graph distances \protect\\ via graph neural networks and node matching}
\maketitle

\begin{abstract}


Computing meaningful distances between graphs is often difficult due to the combinatorial explosion of possible transformations on graphs. Recently researchers started using graph neural networks attempting to use learning methods to replace classical search algorithms. However, these learning methods often have bottlenecks such as fixed size graph embeddings, which prevent them from scaling to larger graphs. We propose to use soft matchings between nodes to overcome this problem. Our novel combination of a classical cost matrix, motivated by bipartite graph matching algorithms, and the modern sinkhorn distance shows state-of-the-art results predicting the ubiquitous graph edit distance. Additionally, we report strong empirical results at graph matching with the same model. Our implementation\footnote{\url{https://gitlab.lrz.de/gdn/graph-distance}} efficiently handles sparse inputs and large real world graphs.

% of pairwise distances between nodes in the form

% However, these learning methods often neglect the

% Graph neural networks suffer bottleneck problems
% We propose to use soft matching between nodes, using a cost matrix, motivated by bipartie graph matching and sinkhorn distance.
% We show empirically that our model improves upon state-of-the-art methods on the task of predicting the graph edit distance
% and demontrate that it can be used for graph matching.
% Our sparsity-aware implementation\footnote{} scales well to large real world graphs.



\end{abstract}

\section{Introduction}

In this report I summarize the work I did during my guided research. The main part of the report is focuses on the findings that are relevant to a potential paper submission, while in the appendix I will write more about the process and some of the less important work I have done.

The main part of the report starts with an analysis of related work in the area of graph distance learning and graph matching. Then \dots

Main contributions:
\begin{itemize}
    \item Propose method for graph distance learing that scales to larger graphs. \\ (With Nystrom/Multiscale $\rightarrow$ First sub $n^2$ that scales to larger graphs).
    \item Experimentaly show that it outperforms other methods on task of Graph Edit Distance estimation.
    \item Show that our application of sinkhorn has advantages over previous work.
\end{itemize}

In summary \dots


\section{Derivation of the model}

% derivation
%     (cost_matrix -> matching) premutaiton invariant (matching is because invariant)

%     GED
%     compare two graphs

%     try matching

%     discrete -> non differentiable
%     -> EMD slow

%     sinkhorn

%     matrix scaling

%     backprop in optimal (Danskin)

%     (nystrom, multiscale sinkhorn)
%     (lanmarks, cluster)

%     GNN
%     Consensus



% (assume E << V^2)
We want our model to be applicable to any set of graphs $\mathcal{G}$, where each graph $G = \{V, E\} \in \mathcal{G}$ with nodes $V$ and edges $E$ is drawn from a common distribution. The nodes and edges may have arbitrary features. Possible distributions include chemical molecules, control flow graphs generated by C++ compilers, or preferential attachment graphs (\citealp{pref_att2002}).

The Graph Edit Distance (GED) between two graphs is defined analog to the String Edit Distance.
\begin{equation}
     \text{GED}(G_{1},G_{2}) = \min_{(e_{1},...,e_{k}) \in \mathcal{P}(g_{1},g_{2})} \sum_{i=1}^{k} c(e_{i})
\end{equation}
where $e_{i}$ are edit operations: edge/node addtion, removal, and substituion, $c(e)$ the cost of an edit operation, and $\mathcal{P}(G_{1},G_{2})$ the edit paths that transform $G_{1}$ into a graph isomorphic to $G_{2}$.

Note that the GED is symetric and invariant to the graph representation. Clearly the GED as defined above cannot depend on the ordering in which the nodes are saved in memory. Moreover, as long as the costs of substituion are symetric, also the GED is symetric ($\text{GED}(G_{1},G_{2}) = \text{GED}(G_{2},G_{1})$). Therefore, a neural network approximating the GED should also be symetric and invariant to the graph representation.

The simplest idea might be to embed both graphs into a common vector space and compute the distance between these two vectors. One can use message passing networks to generate node embeddings. Message passing is invariant to the graph representation and is commom practice in machine learning on graphs due to efficient implemention and great performance. These node embeddings can then be aggregated into a single graph embedding using any aggregation method, possibly combined with an attention layer. Finally, any distance between two vectors (cosine, $p$-norm) will ensure symetric predictions. The main problem with this approach is that the embedding size becomes a bottleneck for larger graphs and the entire model would need to be trained from scratch to increase it.
%  \cite{li2019}, and \cite{bai2019}

To solve the problem we match nodes of the two graphs, similar to classical GED algorithm (\citealp{hungarian2009}; \citealp{frankhauser2011}). These algorithms are based on bipartite graph matching. They set up a cost Matrix $C \in \mathbb{R}^{N \times N}$, where $N = \vert V_1 \vert + \vert V_2 \vert$ and solve the following contrained optimization problem.

% \begin{equation}
%      \begin{gathered}
%           \min \sum_{i = 1}^{N'} \sum_{j = 1}^{N'} T_{ij} C_{ij} \\
%           \text{subject to} \\
%           \sum_{i = 1}^{N'} T_{ij} = 1 \forall j \in \{1 \dots N'\} \\
%           \sum_{j = 1}^{N'} T_{ij} = 1 \forall i \in \{1 \dots N'\} \\
%           T_{ij} \in \{0, 1\}
%      \end{gathered}
% \end{equation}

% \begin{equation}
%      C=
%           \left[
%           \begin{array}{ccc|ccc}
%                C_{1,1} & \dotsi & C_{1, N_2} & C_{1, \epsilon} & \dotsi & \infty \\
%                \vdots & \ddots & \vdots & \vdots & \ddots & \vdots \\
%                C_{N_1, 1} & \dotsi & C_{N_1, N_2} & \infty & \dotsi & C_{N_1, \epsilon} \\
%                \hline
%                C_{\epsilon, 1} & \dotsi & \infty & 0 & \dotsi & 0 \\
%                \vdots & \ddots & \vdots & \vdots & \ddots & \vdots \\
%                \infty & \dotsi & C_{\epsilon, N_2} & 0 & \dotsi & 0 \\
%           \end{array}
%           \right]
% \end{equation}


\noindent
\begin{minipage}{.5\linewidth}

     \[
          C=
               \left[
               \begin{array}{ccc|ccc}
                    C_{1,1} & \dotsi & C_{1, N_2} & C_{1, \epsilon} & \dotsi & \infty \\
                    \vdots & \ddots & \vdots & \vdots & \ddots & \vdots \\
                    C_{N_1, 1} & \dotsi & C_{N_1, N_2} & \infty & \dotsi & C_{N_1, \epsilon} \\
                    \hline
                    C_{\epsilon, 1} & \dotsi & \infty & 0 & \dotsi & 0 \\
                    \vdots & \ddots & \vdots & \vdots & \ddots & \vdots \\
                    \infty & \dotsi & C_{\epsilon, N_2} & 0 & \dotsi & 0 \\
               \end{array}
               \right]
     \]

\end{minipage}%
\begin{minipage}{.5\linewidth}

     \begin{equation}
          \begin{gathered}
               \min \sum_{i = 1}^{N'} \sum_{j = 1}^{N'} T_{ij} C_{ij} \\
               \text{subject to} \\
               \sum_{i = 1}^{N'} T_{ij} = 1 \forall j \in \{1 \dots N'\} \\
               \sum_{j = 1}^{N'} T_{ij} = 1 \forall i \in \{1 \dots N'\} \\
               T_{ij} \in \{0, 1\}
          \end{gathered}
     \end{equation}

\end{minipage}


% \begin{equation}
%      C=
%           \left[
%           \begin{array}{ccc|ccc}
%                C_{1,1} & \dotsi & C_{1, N_2} & C_{1, \epsilon} & \dotsi & C_{1, \epsilon} \\
%                \vdots & \ddots & \vdots & \vdots & \ddots & \vdots \\
%                C_{N_1, 1} & \dotsi & C_{N_1, N_2} & C_{N_1, \epsilon} & \dotsi & C_{N_1, \epsilon} \\
%                \hline
%                C_{\epsilon, 1} & \dotsi & C_{\epsilon, N_2} & 0 & \dotsi & 0 \\
%                \vdots & \ddots & \vdots & \vdots & \ddots & \vdots \\
%                C_{\epsilon, 1} & \dotsi & C_{\epsilon, N_2} & 0 & \dotsi & 0 \\
%           \end{array}
%           \right]
% \end{equation}

% \begin{equation}
%      C=
%           \left[
%           \begin{array}{ccc|ccc}
%                C_{1,1} & \dotsi & C_{1, N_2} & C_{1, \epsilon} \\
%                \vdots & \ddots & \vdots & \vdots \\
%                C_{N_1, 1} & \dotsi & C_{N_1, N_2} & C_{N_1, \epsilon} \\
%           \end{array}
%           \right]
% \end{equation}


Since we don't want to restrict ourselves to the GED. We want to learn. Therefore we use a message passing network to generate node embeddings and compute pairwise distances, yielding a cost matrix like to one above. Then there are multiple paths one could take...





% set up bp cost_matrix

%% From CNN2
% Exactly solving this constrained optimization
% program would yield the exact GED solution
% (Fankhauser, Riesen, and Bunke 2011), but it is NPcomplete since it is equivalent to finding an optimal matching in a complete bipartite graph (Riesen and Bunke 2009).
% To efficiently solve the assignment problem, the Hungarian algorithm (Kuhn 1955) and the Volgenant Jonker (VJ)
% (Jonker and Volgenant 1987) algorithm are commonly used,
% which both run in cubic time. In contrast, GSimCNN takes
% advantage of the exact solutions of the instances of this problem during the training stage, and computes the approximate
% GED during testing in quadratic time, without the need for
% solving any optimization problem for a new graph pair.

% message passing formulas


\section{Related Work}



% approximations riba
% classical methods bai
% our method to compute exact distance
% applications

% Faced with the great significance yet huge difficulty of computing the exact GED between two graphs,
% a flurry of approximate algorithms have been proposed with a trade-off between speed and accuracy.
% However, these methods usually require rather complicated design and implementation based on discrete optimization or combinatorial search. The time complexity is usually polynomial or even sub-exponential in
% the number of nodes in the graphs, such as HED
% (Fischer et al. 2015), Hungarian (Riesen and Bunke 2009),
% VJ (Fankhauser, Riesen, and Bunke 2011), A*-Beamsearch
% (Beam) (Neuhaus, Riesen, and Bunke 2006), etc


The broad range of applicability (of graphs), and the recent success of deep learning across different tasks [vision, rl, nlp] has spurred on the research in the application of deep neural networks to graphs. Borowing the idea of convolutional layers (\citealp{alexnet2012}) from conputer vision, \cite{kipf2017} introduced Graph Convolutional Networks (GCN) which can be implemented efficiently via message passing between connected nodes. Based on this message passing mechanism researchers have improved upon classical state-of-the-art methods in node classification and graph classification [find something in the pytorch-geometric list].

Recently researchers started applying graph neural networks based on the message passing mechanism to the task of GED approximation. Bai et al. proposed in 2018 (published \citealp{bai2019}) to apply the same GCN to both graphs and aggregate the resulting node embeddings with an attention layer into two graph embedding, which can be used to compute a distance. In addition to this fully differentiable and therefore trainable part, they compute a histogram of the pairwise distances between the node embeddings to incorporate some of the local node level information. However, the histogram method is not differentiable and it is unclear if the model can properly exploit this information. Moreover, this process reduces any graph to fixed-size embeddings and is unlikely to scale to larger graphs without hand-tuning the embedding size. Despite these flaws the authors show that their method significantly outperforms classical methods while being orders of magnitude faster at inference.
% In particular they compared their results with the Hungarian algorithm (\citealp{hungarian2009}), VJ

% The first category of baselines includes three classic algorithms
% for GED computation. (1) A*-Beamsearch (Beam) [28]. It is a variant
% of the A* algorithm in sub-exponential time. (2) Hungarian [23, 35]
% and (3) Volgenant Jonker VJ [9, 18] are two cubic-time algorithms based on the Hungarian Algorithm for bipartite graph matching, and the algorithm
% of Volgenant and Jonker, respectively

Later \cite{bai2018_cnn1} proposed to compute the pairwise distances between nodes after each layer of the GCN and apply standard 2-D Convolutional Neural Networks (CNN) to these matrices. The main problems with this approach as described in their work are:
\begin{itemize}
     \item \textbf{Permutation invariance.} The ordering of the nodes in the graph represenation should not change the computed distance.
     \item \textbf{Spatial locality.} CNN's introduce a strong strucutral prior by making the assumption that nearby points are strongly realted while further apart points are not.
\end{itemize}
The authors alleviate these problems by using breadth-first-search node-ordering (\citealp{bfs2018}). They claim that nearby nodes are sorted close to each other. However, clearly this cannot be true in general for all nodes. Furthermore, the ordering is not unique, as it is not yet known if there exists an polynomial algorithm to find a canonical ordering (\citealp{canonical2016}). While this is not a principled approach the authors show an intersting connection between convolutional kernels and the optimal assignment (Section 4.1) and show strong empirical results compared to the previously mentioned classical algorithms.
% and bipartite matching (Section 4.2)

Presumalby indepently, \cite{riba2018} proposed a similar siamese network to estimate graph distances.  They apply the same GCN to both graphs generating sets $A$ and $B$ of node embeddings (one for each graph). The distance between two graphs is then computed as:
\begin{equation}
     d(g_{1}, g_{2}) = \frac{\sum_{a \in A} \inf_{b \in B} d(a, b) + \sum_{a \in A} \inf_{b \in B} d(a, b)}{\vert V_1 \vert + \vert V_2 \vert}
\end{equation}
This equation is derived as a soft variant of the Hausdorf Distance, which is also known as the Chamfer Distance in computer vision (\citealp{chamfer1977}). Instead of approximating the GED directly they validate their method on graph classification and keyword spotting datasets. Therefore, their empirical results are difficult to compare to Bai et al. (2018; 2019). This method is interesting because it generates a complete matching between nodes and might therefore scale to larger graphs without hand-tuning the embedding size. Local node structures are incorporated during the message passing, and global and local structures are relevant for the final distance.

\cite{li2019} used a similar method, called Graph Matching Network (GMN), for graph classification and detecting vulneralbe binaries by comparing control flow graphs of different compilers. The GMN also employs message passing layers to both graphs. However, they modify the message passing to allow information of one graph to flow into the node updates of the other graph. % TODO? cross-graph attention formulas
This process is fully differentiable and makes use of local node level and global graph level information but still suffers from the problem of reducing the entire graph to a single embedding vector which is expected to scale unfavourably to larger graphs. Moreover, computing the cross-graph matching vector costs $O(\vert V_1 \vert \vert V_2 \vert)$. While all previously described models have technically the same complexity ($O(\vert V_1 \vert \vert V_2 \vert + \vert E_1 \vert + \vert E_2 \vert)$), the GMN computes this cross-graph matching vector for each layer.

Note that in principle one could also apply GMN layers and then use the Soft Hausdorf Distance, making these two methods somewhat orthogonal.

\cite{fey2020_update} apply the GCN to graph matching, meaning that the network predicts pairs of nodes that correspond to each other. In particular they test their method on keypoint matching of objects in different pictures and cross-lingual knowledge graph alignment. Although this is different task their model is extremly similar to the one of \cite{riba2018}. They also apply the same GCN to both graphs (siamse network) and compute a matching. Instead of the Soft Hausdorf Distance \cite{fey2020_update} use the Sinkhorn Nomralization \cite{sinkhorn2013} on the matrix of pairwise distances. The Sinkhorn Nomralization is an iterative process that returns a doubly stochastic matrix, which is interpreted as a soft correspondence between nodes. One could immediately find a matching by taking the maximum likelihood estimate, but the authors then apply their proposed consensus steps. They generate an injective node coloring (or random node embeddings in practice) for one graph and use the soft correspondence matrix to map these embeddings to the other graph. Using the generated node embeddings they apply another GCN to both graphs yielding, after nomralization, a new soft correspondence matrix. This consensus step is repeated multiple times and the final answer is found from the last soft correspondence matrix. Unfortunately, the authors report that in practice they used the Softmax Operator on each row, instead of Sinkhorn Normalizaton due to better gradients. This can be a problem if the graphs have different number of nodes. In a toy experiment they show that with consensus steps their method is robust to node additions and removal, but it certainly is less principled because the soft corrospondence matrix and the resulting matching is not necessarily symmetric, $M(G_1, G_2) \neq M(G_2, G_1)$.

% consensus is orthogonal to our work


% maybe add how everyone computes thier pairwise distances
% < a, b >, MLP(a - b), ||a - b||_p


% In the end sinkhorn > soft hausdorf, and gmn is probably obsolete
% really fast only if we never genreate a full matching


\section{Setup}

Siamese GCN.
Set up cost matrix (bp vs. normal, scale with $\alpha$).
Compoute explicit matching with sinkhorn (use cost matrix as kernel, can tune $\lambda$).
Use optimal matching for backprop.
Multiply matching with cost matrix.



% Our message function
\begin{equation}
     h_i^{l} = \sigma(W_{l} h_i^{l-1}) + \sum_{j \rightarrow i} \sigma(W_{l} h_j^{l-1}) E_{e_{j \rightarrow i}}
\end{equation}
norms



% \begin{equation}
%      C=
%           \left[
%           \begin{array}{ccc|ccc}
%                C_{1,1} & \dotsi & C_{1, N_2} & C_{1, \epsilon} & \dotsi & C_{1, \epsilon} \\
%                \vdots & \ddots & \vdots & \vdots & \ddots & \vdots \\
%                C_{N_1, 1} & \dotsi & C_{N_1, N_2} & C_{N_1, \epsilon} & \dotsi & C_{N_1, \epsilon} \\
%                \hline
%                C_{\epsilon, 1} & \dotsi & C_{\epsilon, N_2} & 0 & \dotsi & 0 \\
%                \vdots & \ddots & \vdots & \vdots & \ddots & \vdots \\
%                C_{\epsilon, 1} & \dotsi & C_{\epsilon, N_2} & 0 & \dotsi & 0 \\
%           \end{array}
%           \right]
% \end{equation}

% \begin{equation}
%      C=
%           \left[
%           \begin{array}{ccc|ccc}
%                C_{1,1} & \dotsi & C_{1, N_2} & C_{1, \epsilon} \\
%                \vdots & \ddots & \vdots & \vdots \\
%                C_{N_1, 1} & \dotsi & C_{N_1, N_2} & C_{N_1, \epsilon} \\
%           \end{array}
%           \right]
% \end{equation}


\section{Results}

\subsection{Experiment 1}

GED: Us vs. Riba, Bai, Li

Describe all the possible settings, the best settings, and more. As you can see in Table \ref{tab:ex1-baselines} and in Table \ref{tab:ex1-ablation} there are many things happening.


\begin{table}[htbp]
    \addtolength{\tabcolsep}{-1pt}
    \fontsize{9pt}{10.25pt}\selectfont
    \centering
    \renewcommand{\arraystretch}{1.2}
    \begin{tabular}{|l|c|c|c|c|}
        \hline
        \multirow{2}{*}{} & \multicolumn{2}{c|}{Pref-Attachment} & \multicolumn{2}{c|}{Aids} \\ \hhline{|~|-|-|-|-|}
        & Val & Test & Val & Test \\ \hhline{|=|=|=|=|=|}
        Riba et al. & $5.5 \pm 0.3$ & $5.5 \pm 0.3$ & $5.5 \pm 0.3$ & $5.5 \pm 0.3$ \\ \hline
        Bai et al. & X & X & Y & Y \\ \hline
        Li et al. & X & X & Y & Y \\ \hline
        GDN & X & X & Y & Y \\ \hline
    \end{tabular}
    \caption{GED Experiment}
    \label{tab:ex1-baselines}
\end{table}


\begin{table}[htbp]
    \addtolength{\tabcolsep}{-1pt}
    \fontsize{9pt}{10.25pt}\selectfont
    \centering
    \renewcommand{\arraystretch}{1.2}
    \begin{tabular}{|c|c|c|c|c|c|}
        \hline
        \multicolumn{2}{|c|}{} & \multicolumn{2}{c|}{Pref-Attachment} & \multicolumn{2}{c|}{Aids} \\ \hline
        Bp-Dist & Norm & Val & Test & Val & Test \\ \hhline{|=|=|=|=|=|=|}
        \multirow{3}{*}{yes} & $p=1$ & $5.5 \pm 0.3$ & $5.5 \pm 0.3$ & $5.5 \pm 0.3$ & $5.5 \pm 0.3$ \\ \hhline{|~|-|-|-|-|-|}
        & $p=2$ & X & X & Y & Y \\ \hhline{|~|-|-|-|-|-|}
        & MLP & X & X & Y & Y \\ \hline
        \multirow{3}{*}{no}  & $p=1$ & $5.5 \pm 0.3$ & $5.5 \pm 0.3$ & $5.5 \pm 0.3$ & $5.5 \pm 0.3$ \\ \hhline{|~|-|-|-|-|-|}
        & $p=2$ & X & X & Y & Y \\ \hhline{|~|-|-|-|-|-|}
        & MLP & X & X & Y & Y \\ \hline
    \end{tabular}
    \caption{Ablation Study}
    \label{tab:ex1-ablation}
\end{table}

% pref_att
% riba (running)
% bai
% gmn
% gdn

% aids
% riba (remove some settings!)
% bai (running)
% gmn
% gdn

% Note bai has those 115 / 54 mse's


\subsection{Experiment 2}

Keypoint matching vs. Fey




\begin{table}[ht]
    \centering
    \resizebox{\textwidth}{!}{\begin{tabular}{ccccccccccccccccccccccc}
        \hline
        \textbf{Method}& &\textbf{Aero}&\textbf{Bike}&\textbf{Bird}&\textbf{Boat}&\textbf{Bottle}&\textbf{Bus}&\textbf{Car}&\textbf{Cat}&\textbf{Chair}&\textbf{Cow}&\textbf{Table}&\textbf{Dog}&\textbf{Horse}&\textbf{M-Bike}&\textbf{Person}&\textbf{Plant}&\textbf{Sheep}&\textbf{Sofa}&\textbf{Train}&\textbf{TV}&\textbf{Mean} \\
        \hline
        DGC & L=0 &42.1&57.5&49.6&59.4&83.8&84.0&78.4&67.5&37.3&60.4&85.0&58.0&66.0&54.1&52.6&93.9&60.2&85.6&87.8&82.5&67.3\\
        \hline
        GDN & L=0&45.3&63.8&55.5&72.6&88.9&88.5&79.5&69.1&36.8&58.5&82.8&59.2&70.1&59.8&53.4&96.3&63.7&92.5&91.7&85.0&70.5\\
        \hline
        DGC & L=10&45.5&67.6&56.5&66.8&86.9&85.2&84.2&73.0&43.6&66.0&92.3&64.0&79.8&56.6&56.1&95.4&64.4&95.0&91.3&86.3&72.8\\
        \hline
    \end{tabular}}
    \caption{Hits@1 (\%) on the PASCALVOC dataset with Berkeley keypoint annotations}
\end{table}


% "config" : {
%     "alpha_vector" : true,
%     "batch_size" : 512,
%     "bp_dist_matrix" : false,
%     "consensus_type" : "none",
%     "dataname" : "pascal",
%     "db_collection" : "gdn_match_report_test2",
%     "device" : "cuda",
%     "emb_dist_p" : 2,
%     "gnn_type" : "spline",
%     "graph_distance" : "matching",
%     "learning_rate" : 0.01,
%     "max_epochs" : 100,
%     "model_type" : "gdn",
%     "overwrite" : 45,
%     "patience" : 5,
%     "run" : 0,
%     "seed" : 318116669,
%     "sinkhorn_reg" : 0.1,
%     "sparse_batching" : false
% },
% Epoch 13/99,    train    loss: 11.3083, hits@1: 0.7304 (5.90s)
% 2020-03-04 12:50:46 (INFO): Epoch 13/99,    val      0.4341 0.5750 0.6551 0.6753 0.8093 0.7804 0.7854 0.6490 0.4232 0.6160 0.8913 0.6165 0.6332 0.6561 0.6108 0.9710 0.6804 0.6545 0.8833 0.8951
% 2020-03-04 12:50:46 (INFO): Epoch 13/99,    val      loss: 13.9017, hits@1: 0.6948 (3.39s)
% 2020-03-04 12:50:49 (INFO): Epoch 13/99,    test     0.4533 0.6383 0.5547 0.7259 0.8890 0.8506 0.7945 0.6909 0.3682 0.5850 0.8276 0.5924 0.7071 0.5984 0.5341 0.9634 0.6371 0.9254 0.9171 0.8500
% 2020-03-04 12:50:49 (INFO): Epoch 13/99,    test     loss: 13.2461, hits@1: 0.7052 (3.38s)

% 0.4533 0.6383 0.5547 0.7259 0.8890 0.8506 0.7945 0.6909 0.3682 0.5850 0.8276 0.5924 0.7071 0.5984 0.5341 0.9634 0.6371 0.9254 0.9171 0.8500
% 0.7052



\section{Conclusion}

Using matching for distance computation has great potential. Can compute matching fast and still be fully differentiable. Works great on GED.

% improves upon state-of-the-art methods in predicting the GED
% no reason why we couldn't apply it to other distances


\section{Outlook}

The first problem that should be tackled is the gap between validation and test results on the AIDS dataset. Increasing the number of graphs could potentially solve the issue already. However, to verify any improvement we should also consider working with multiple completely distinct validation sets at once.

Clearly, also the underperforming consensus method should be improved in the future. We are able to produce a solid initial matching, and simply applying the same "softmax-no-norms" consensus of \cite{fey2020_update} should yield at least as good results as they report. Therefore, exactly replicating their model might be a good start and allow us to then investigate which modifications can further improve these results.

After that we might want to look back the four defining properties of metrics:
\begin{itemize}
    \itemsep0em
    \item $d(x,y) \ge 0$ non-negativity
    \item $d(x,y) = 0 \Leftrightarrow x = y$ identity of indiscernibles
    \item $d(x,y)  = d(y,x)$ symmetry
    \item $d(x,y) \le d(x,z) + d(z, y)$ subadditivity or triangle inequality
\end{itemize}
We already ensured non-negativity, and symmetry, but it would certainly be interesting to investigate if we can ensure the triangle inequality. Perhaps that could lead to a small modification which induces a strong and useful prior. Also investigating the identity of indiscernibles could lead to interesting results. If it was possible to ensure this property, it would solve the graph isomorphism problem, and finally answer the question of whether the problem is NP-complete or not. However, doing so will require the use of more powerful embedding networks because a standard GNN cannot distinguish between certain graphs\footnote{For example, a GNN cannot distinguish between a hexagon and two disconnected triangles.} and will therefore yield exactly the same set of embeddings in such a case. Maybe easier to handle and therefore more promising is translational invariance, $d(x,y) = d(x+a,y+a)$, which might be a valid assumptions in many cases. Graph sums can be defined by adding the adjacency matrices of equal sized graphs \cite{graph_sum2004} or, in the context of the graph edit distance, maybe by adding a node that is connected to all other nodes. If it is not possible to derive enforcing model modifications for translational invariance or the triangle inequality then one could still try using the properties for data augmentation, which could certainly lead to better generalization to the test set.

A different direction for further research is to speed up the current model. It seems inevitable that the whole process is quadratic in the number of nodes due to the use of the cost matrix $C \in \mathbb{R}^{N \times N}$. However, we might be able to achieve linear run time by using the Nystr{\"{o}}m method or multiscale matching. \cite{nytrom2019} show that using the Nystr{\"{o}}m method and Sinkhorn scaling one can accurately approximate the Sinkhorn distance without ever computing the full cost matrix. The Nystr{\"{o}}m method is based on a set of landmarks that is then used to speed up matching. Similarly multiscale matching uses clusters of nodes that are then matched with each other (\citealp{multiscale2016}). One problem with both approaches is that they only produce the final distance and not an explicit matching or a complete cost matrix, which would be required to apply Danskin's theorem for fast backpropagation. Therefore, the key for implementing both of these methods will be to find an efficient way to backpropagate the gradients. Ideally one would combine both approaches since landmarks and clusters each have their merits and demerits. Using landmarks one can vastly overestimate small distances, and multiscale matching never computes distances which exceed some threshold.

To show off the advantage of our full BP cost matrix it would be interesting to investigate "sparse matchings". \cite{fey2020_update} use only pairs of graphs where every node in the source graph finds a match in the target graph. We could simply add the other pairs to the training set, but validate on both groups individually. There should be a good chance that we significantly outperform any previous methods on this specialized task.


\newpage

\bibliographystyle{plainnat}
\bibliography{egbib}

% cite    -> Name (year)
% citealp -> Name, year

\newpage



\appendix
\appendixpage


Here the proof of Theorem \ref{theorem} and some more practical findings, which are not immediately relevant to the main story of this report:

\section{Proof of Theorem \ref{theorem}} \label{proof}

\begin{proof}
Let $B_R = \{f \in \mathscr{H}, \norm{f}_{\mathscr{H}} < R\} \subset \mathscr{H}$ be the ball of functions which have a RKHS norm less than or equal to $R$, let $l(f(\mathbf{x}), y) = \max(t - y f(\mathbf{x}), 0)$ be the hinge loss with margin $t$, and let $V_{\gamma}(B_R)$ be the fat shattering dimension of the function space $B_R$ with the parameter $\gamma$.

Then by Anthony and Bartlett \cite{bartlett2002} $\exists C_1, C_2 > 0$ such that with high probability $\forall_{f \in B_R}$:
\begin{equation}
    \abs{ \frac{1}{n} \sum_i l(f(\mathbf{x}_i), y_i) - \mathbb{E}_P[l(f(\mathbf{x}), y)] } \leq C_1 \gamma + C_2 \sqrt{\frac{V_{\gamma}(B_R)}{n}}
\end{equation}

Since $y$ is not a deterministic function of $x$, $\mathbb{E}_P[l(f(\mathbf{x}), y)] > 0$. Now we fix $\gamma > 0$ such that $C_1 \gamma < \mathbb{E}_P[l(f(\mathbf{x}), y)]$. And suppose $h \in B_R$ $t$-overfits the data, then by construction $\frac{1}{n} \sum_i l(h(\mathbf{x}_i), y_i) = 0$.

\begin{equation}
     0 < \mathbb{E}_P[l(f(\mathbf{x}), y)] - C_1 \gamma <  C_2 \sqrt{\frac{V_{\gamma}(B_R)}{n}}
\end{equation}
\begin{equation}
    \frac{n}{C_2^2}(\mathbb{E}_P[l(f(\mathbf{x}), y)] - C_1 \gamma)^2 <  V_{\gamma}(B_R)
\end{equation}

Then by Belkin \cite{belkin2018b} $V_{\gamma}(B_R) < O(\log^d(\frac{R}{\gamma}))$, meaning that:

\begin{equation}
    \frac{n}{C_2^2}(\mathbb{E}_P[l(f(\mathbf{x}), y)] - C_1 \gamma)^2 < \log^d(\frac{R}{\gamma})
\end{equation}
\begin{equation}
     A \mathrm{e}^{B n^\frac{1}{d}} < R \mbox{ where } A, B > 0
\end{equation}

\end{proof}




\section{Early Stopping would have helped}

\begin{figure}[!htb]
    \centering
    % \includegraphics[width=0.49\linewidth]{img/Overfit_Synthetic22.png}
    \caption{Classification error over epochs on Synthetic dataset.}
    \label{fig:stopping}
\end{figure}

During the second experiment we trained the classifiers until (almost) zero classification error since that is required for Theorem \ref{theorem} to hold. However it is very interesting that in Figure \ref{fig:stopping} we can actually see the classifier using the Gaussian Kernel drops in performance after only a few iterations on the Sythetic dataset. Recall, that this is the different to our results on MNIST and CIFAR-10.



\section{Computational Reach}

\begin{figure}[!htb]
\begin{subfigure}{\textwidth}
    \centering
         \begin{tabular}{|c | c | c | c | c |}
         \hline
         Label & MNIST & CIFAR-10 & Synthetic 1 & Synthetic 2\\ [0.5ex]
         \hline
         Original & 8 & $>$300 & 1 & 205 \\
         \hline
         Random & $>$300 & $>$300 & 106 & 269 \\
         \hline
         \end{tabular}
\caption{Gaussian Kernel} \label{fig:ta}
\end{subfigure}
\\
\vspace{0.2cm}
\\
\begin{subfigure}{\textwidth}
    \centering
         \begin{tabular}{|c | c | c | c | c |}
         \hline
         Label & MNIST & CIFAR-10 & Synthetic 1 & Synthetic 2\\ [0.5ex]
         \hline
         Original & 3 & 5 & 1 & 5 \\
         \hline
         Random & 7 & 5 & 4 & 4 \\
         \hline
         \end{tabular}
\caption{Laplacian Kernel} \label{fig:tb}
\end{subfigure}
\caption{Number of Epochs until training reached zero classification error.} \label{fig:table}
\end{figure}

In Figure \ref{fig:stopping} the Laplacian Kernel strongly outperforms the Gaussian in terms of test error, and that follows a general trend visible in all our experiments. Also very interesting in that regard is the high computational reach of the Laplacian Kernel (confer to Figure \ref{fig:table}), which is similar to the findings of Zhang et al. \cite{zhang2017} using neural networks with ReLU units. Therefore the Laplacian Kernel is not only better in terms of performance in our experiments, but also vastly superior in terms of computational effort.



\section{Bandwidth}

\begin{figure}[!htb]
\centering
\begin{subfigure}{0.42\textwidth}
% \includegraphics[width=\linewidth]{img/Figure8_Gaussian.png}
\caption{Gauss} \label{fig:bsa}
\end{subfigure}
\hspace*{\fill} % separation between the subfigures
\begin{subfigure}{0.42\textwidth}
% \includegraphics[width=\linewidth]{img/Figure8_Laplace.png}
\caption{Laplace} \label{fig:bb}
\end{subfigure}
\caption{Comparison between gauss, laplace.} \label{fig:bandwidth}
\end{figure}

Another interesting experiment can be seen in Figure \ref{fig:bandwidth}. Here we can see that smaller bandwidth (i.e. less smoothness in case of the Gaussian Kernel) results in better performance for higher levels of noise.  This gives us the intuition that maybe the possibility for sharper "cut outs" helps generalizing on noisy datasets.



\end{document}
