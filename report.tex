\documentclass[a4paper,10pt]{article}
\usepackage[utf8]{inputenc}
\usepackage[margin=3cm]{geometry}
\usepackage{amsmath}
\usepackage{amssymb}
\usepackage{amsthm}
\usepackage{fancyhdr}
\usepackage{seminar}
\usepackage{graphicx}
% \usepackage{subfigure}
\usepackage{float}
\usepackage{hyperref}


\usepackage{mathrsfs}
\usepackage{commath}
\usepackage{mathtools}
\usepackage{subcaption}
\usepackage{appendix}

\DeclareMathOperator*{\argmax}{arg\,max}
\DeclareMathOperator*{\argmin}{arg\,min}

\pagestyle{fancy}


%You can add theorem-like environments (e.g. remark, definition, ...) if you want
\newtheorem{theorem}{Theorem}

\title{Kernel methods and Deep Learning II} % Replace with your title
\author{Johannes Pitz} % Replace with your name
\institute{\textit{Seminar: Optimization and Generalization in Deep Learning}}

\makeatletter
\let\runauthor\@author
\let\runtitle\@title
\makeatother
\lhead{\runauthor}
\rhead{\runtitle}


\begin{document}

\maketitle

\begin{abstract}

In deep neural networks often models which were trained until they reach zero training error with little to no regularization perform better on unseen test sets than the same model trained with early stopping and/or explicit regularization. This effect is contradicting the general wisdom taught in machine learning classes, that overfitting will lead to decreasing test performance. It is not well understood, why these generalization properties can be observed. 

This report will show that the mentioned effect is not unique to deep neural networks, but can also be observed in kernel methods. We will show that kernel methods can be viewed as two-layer neural networks. Therefore it might be advantageous to closely investigate the much simpler kernel methods before trying to understand more complex deep architectures. Moreover, we argue that the effect might not be a property of the model alone, but instead depend on the data, optimizer and model.

% (127 words) + last sentence

\end{abstract}

\section{Introduction}

In this report I summarize the work I did during my guided research. The main part of the report is focuses on the findings that are relevant to a potential paper submission, while in the appendix I will write more about the process and some of the less important work I have done.

The main part of the report starts with an analysis of related work in the area of graph distance learning and graph matching. Then \dots

Main contributions:
\begin{itemize}
    \item Propose method for graph distance learing that scales to larger graphs. \\ (With Nystrom/Multiscale $\rightarrow$ First sub $n^2$ that scales to larger graphs).
    \item Experimentaly show that it outperforms other methods on task of Graph Edit Distance estimation.
    \item Show that our application of sinkhorn has advantages over previous work.
\end{itemize}

In summary \dots


\section{Related Work}

Except from the mainly analyzed work by Belkin et al. \cite{belkin2018a} most research regarding the surprising generalization properties of overfitted classifiers is done in deep neural networks. For example, Poggio et al. \cite{poggio2018} showed that properties of stochastic gradient descent (SGD) for linear networks also hold for deep non-linear networks. Firstly, SGD enforces an implicit form of regularization and converges to the minimum norm solution, and secondly for classification tasks this minimum norm solution is also the maximum margin solution. Therefore, it yields good generalization results on low noise datasets. Bartlett et al. \cite{bartlett2017} showed that SGD selects predictors whose complexity scales with the difficulty of the learning task and conclude that the good generalization is a property of the optimization algorithm. Zhang et al. \cite{zhang2017} successfully trained state-of-the-art convolutions neural networks (CNN) on randomly labeled datasets close to zero classification error and found that choosing the right model family or regularization techniques doesn't explain the small difference between training and test performance on correctly labeled datasets.


\section{Approach}

Siamese GCN.
Set up cost matrix (bp vs. normal, scale with $\alpha$).
Compoute explicit matching with sinkhorn (use cost matrix as kernel, can tune $\lambda$).
Use optimal matching for backprop.
Multiply matching with cost matrix.


%% From CNN2
% Exactly solving this constrained optimization
% program would yield the exact GED solution
% (Fankhauser, Riesen, and Bunke 2011), but it is NPcomplete since it is equivalent to finding an optimal matching in a complete bipartite graph (Riesen and Bunke 2009).
% To efficiently solve the assignment problem, the Hungarian algorithm (Kuhn 1955) and the Volgenant Jonker (VJ)
% (Jonker and Volgenant 1987) algorithm are commonly used,
% which both run in cubic time. In contrast, GSimCNN takes
% advantage of the exact solutions of the instances of this problem during the training stage, and computes the approximate
% GED during testing in quadratic time, without the need for
% solving any optimization problem for a new graph pair.


\section{Results}

\subsection{Experiment 1}

GED: Us vs. Riba, Bai, Li

Describe all the possible settings, the best settings, and more. As you can see in Table \ref{tab:ex1-baselines} and in Table \ref{tab:ex1-ablation} there are many things happening.


\begin{table}[htbp]
    \addtolength{\tabcolsep}{-1pt}
    \fontsize{9pt}{10.25pt}\selectfont
    \centering
    \renewcommand{\arraystretch}{1.2}
    \begin{tabular}{|l|c|c|c|c|}
        \hline
        \multirow{2}{*}{} & \multicolumn{2}{c|}{Pref-Attachment} & \multicolumn{2}{c|}{Aids} \\ \hhline{|~|-|-|-|-|}
        & Val & Test & Val & Test \\ \hhline{|=|=|=|=|=|}
        Riba et al. & $5.5 \pm 0.3$ & $5.5 \pm 0.3$ & $5.5 \pm 0.3$ & $5.5 \pm 0.3$ \\ \hline
        Bai et al. & X & X & Y & Y \\ \hline
        Li et al. & X & X & Y & Y \\ \hline
        GDN & X & X & Y & Y \\ \hline
    \end{tabular}
    \caption{GED Experiment}
    \label{tab:ex1-baselines}
\end{table}


\begin{table}[htbp]
    \addtolength{\tabcolsep}{-1pt}
    \fontsize{9pt}{10.25pt}\selectfont
    \centering
    \renewcommand{\arraystretch}{1.2}
    \begin{tabular}{|c|c|c|c|c|c|}
        \hline
        \multicolumn{2}{|c|}{} & \multicolumn{2}{c|}{Pref-Attachment} & \multicolumn{2}{c|}{Aids} \\ \hline
        Bp-Dist & Norm & Val & Test & Val & Test \\ \hhline{|=|=|=|=|=|=|}
        \multirow{3}{*}{yes} & $p=1$ & $5.5 \pm 0.3$ & $5.5 \pm 0.3$ & $5.5 \pm 0.3$ & $5.5 \pm 0.3$ \\ \hhline{|~|-|-|-|-|-|}
        & $p=2$ & X & X & Y & Y \\ \hhline{|~|-|-|-|-|-|}
        & MLP & X & X & Y & Y \\ \hline
        \multirow{3}{*}{no}  & $p=1$ & $5.5 \pm 0.3$ & $5.5 \pm 0.3$ & $5.5 \pm 0.3$ & $5.5 \pm 0.3$ \\ \hhline{|~|-|-|-|-|-|}
        & $p=2$ & X & X & Y & Y \\ \hhline{|~|-|-|-|-|-|}
        & MLP & X & X & Y & Y \\ \hline
    \end{tabular}
    \caption{Ablation Study}
    \label{tab:ex1-ablation}
\end{table}

% pref_att
% riba (running)
% bai
% gmn
% gdn

% aids
% riba (remove some settings!)
% bai (running)
% gmn
% gdn

% Note bai has those 115 / 54 mse's


\subsection{Experiment 2}

Keypoint matching vs. Fey




\begin{table}[ht]
    \centering
    \resizebox{\textwidth}{!}{\begin{tabular}{ccccccccccccccccccccccc}
        \hline
        \textbf{Method}& &\textbf{Aero}&\textbf{Bike}&\textbf{Bird}&\textbf{Boat}&\textbf{Bottle}&\textbf{Bus}&\textbf{Car}&\textbf{Cat}&\textbf{Chair}&\textbf{Cow}&\textbf{Table}&\textbf{Dog}&\textbf{Horse}&\textbf{M-Bike}&\textbf{Person}&\textbf{Plant}&\textbf{Sheep}&\textbf{Sofa}&\textbf{Train}&\textbf{TV}&\textbf{Mean} \\
        \hline
        DGC & L=0 &42.1&57.5&49.6&59.4&83.8&84.0&78.4&67.5&37.3&60.4&85.0&58.0&66.0&54.1&52.6&93.9&60.2&85.6&87.8&82.5&67.3\\
        \hline
        GDN & L=0&45.3&63.8&55.5&72.6&88.9&88.5&79.5&69.1&36.8&58.5&82.8&59.2&70.1&59.8&53.4&96.3&63.7&92.5&91.7&85.0&70.5\\
        \hline
        DGC & L=10&45.5&67.6&56.5&66.8&86.9&85.2&84.2&73.0&43.6&66.0&92.3&64.0&79.8&56.6&56.1&95.4&64.4&95.0&91.3&86.3&72.8\\
        \hline
    \end{tabular}}
    \caption{Hits@1 (\%) on the PASCALVOC dataset with Berkeley keypoint annotations}
\end{table}


% "config" : {
%     "alpha_vector" : true,
%     "batch_size" : 512,
%     "bp_dist_matrix" : false,
%     "consensus_type" : "none",
%     "dataname" : "pascal",
%     "db_collection" : "gdn_match_report_test2",
%     "device" : "cuda",
%     "emb_dist_p" : 2,
%     "gnn_type" : "spline",
%     "graph_distance" : "matching",
%     "learning_rate" : 0.01,
%     "max_epochs" : 100,
%     "model_type" : "gdn",
%     "overwrite" : 45,
%     "patience" : 5,
%     "run" : 0,
%     "seed" : 318116669,
%     "sinkhorn_reg" : 0.1,
%     "sparse_batching" : false
% },
% Epoch 13/99,    train    loss: 11.3083, hits@1: 0.7304 (5.90s)
% 2020-03-04 12:50:46 (INFO): Epoch 13/99,    val      0.4341 0.5750 0.6551 0.6753 0.8093 0.7804 0.7854 0.6490 0.4232 0.6160 0.8913 0.6165 0.6332 0.6561 0.6108 0.9710 0.6804 0.6545 0.8833 0.8951
% 2020-03-04 12:50:46 (INFO): Epoch 13/99,    val      loss: 13.9017, hits@1: 0.6948 (3.39s)
% 2020-03-04 12:50:49 (INFO): Epoch 13/99,    test     0.4533 0.6383 0.5547 0.7259 0.8890 0.8506 0.7945 0.6909 0.3682 0.5850 0.8276 0.5924 0.7071 0.5984 0.5341 0.9634 0.6371 0.9254 0.9171 0.8500
% 2020-03-04 12:50:49 (INFO): Epoch 13/99,    test     loss: 13.2461, hits@1: 0.7052 (3.38s)

% 0.4533 0.6383 0.5547 0.7259 0.8890 0.8506 0.7945 0.6909 0.3682 0.5850 0.8276 0.5924 0.7071 0.5984 0.5341 0.9634 0.6371 0.9254 0.9171 0.8500
% 0.7052



\section{Conclusion}

Using matching for distance computation has great potential. Can compute matching fast and still be fully differentiable. Works great on GED.

% improves upon state-of-the-art methods in predicting the GED
% no reason why we couldn't apply it to other distances


\bibliographystyle{plain}
\bibliography{egbib}

\newpage



\appendix
\appendixpage


Here the proof of Theorem \ref{theorem} and some more practical findings, which are not immediately relevant to the main story of this report:

\section{Proof of Theorem \ref{theorem}} \label{proof}

\begin{proof}
Let $B_R = \{f \in \mathscr{H}, \norm{f}_{\mathscr{H}} < R\} \subset \mathscr{H}$ be the ball of functions which have a RKHS norm less than or equal to $R$, let $l(f(\mathbf{x}), y) = \max(t - y f(\mathbf{x}), 0)$ be the hinge loss with margin $t$, and let $V_{\gamma}(B_R)$ be the fat shattering dimension of the function space $B_R$ with the parameter $\gamma$.

Then by Anthony and Bartlett \cite{bartlett2002} $\exists C_1, C_2 > 0$ such that with high probability $\forall_{f \in B_R}$:
\begin{equation}
    \abs{ \frac{1}{n} \sum_i l(f(\mathbf{x}_i), y_i) - \mathbb{E}_P[l(f(\mathbf{x}), y)] } \leq C_1 \gamma + C_2 \sqrt{\frac{V_{\gamma}(B_R)}{n}}
\end{equation}

Since $y$ is not a deterministic function of $x$, $\mathbb{E}_P[l(f(\mathbf{x}), y)] > 0$. Now we fix $\gamma > 0$ such that $C_1 \gamma < \mathbb{E}_P[l(f(\mathbf{x}), y)]$. And suppose $h \in B_R$ $t$-overfits the data, then by construction $\frac{1}{n} \sum_i l(h(\mathbf{x}_i), y_i) = 0$.

\begin{equation}
     0 < \mathbb{E}_P[l(f(\mathbf{x}), y)] - C_1 \gamma <  C_2 \sqrt{\frac{V_{\gamma}(B_R)}{n}}
\end{equation}
\begin{equation}
    \frac{n}{C_2^2}(\mathbb{E}_P[l(f(\mathbf{x}), y)] - C_1 \gamma)^2 <  V_{\gamma}(B_R)
\end{equation}

Then by Belkin \cite{belkin2018b} $V_{\gamma}(B_R) < O(\log^d(\frac{R}{\gamma}))$, meaning that:

\begin{equation}
    \frac{n}{C_2^2}(\mathbb{E}_P[l(f(\mathbf{x}), y)] - C_1 \gamma)^2 < \log^d(\frac{R}{\gamma})
\end{equation}
\begin{equation}
     A \mathrm{e}^{B n^\frac{1}{d}} < R \mbox{ where } A, B > 0
\end{equation}

\end{proof}




\section{Early Stopping would have helped}

\begin{figure}[!htb]
    \centering
    % \includegraphics[width=0.49\linewidth]{img/Overfit_Synthetic22.png}
    \caption{Classification error over epochs on Synthetic dataset.}
    \label{fig:stopping}
\end{figure}

During the second experiment we trained the classifiers until (almost) zero classification error since that is required for Theorem \ref{theorem} to hold. However it is very interesting that in Figure \ref{fig:stopping} we can actually see the classifier using the Gaussian Kernel drops in performance after only a few iterations on the Sythetic dataset. Recall, that this is the different to our results on MNIST and CIFAR-10.



\section{Computational Reach}

\begin{figure}[!htb]
\begin{subfigure}{\textwidth}
    \centering
         \begin{tabular}{|c | c | c | c | c |}
         \hline
         Label & MNIST & CIFAR-10 & Synthetic 1 & Synthetic 2\\ [0.5ex]
         \hline
         Original & 8 & $>$300 & 1 & 205 \\
         \hline
         Random & $>$300 & $>$300 & 106 & 269 \\
         \hline
         \end{tabular}
\caption{Gaussian Kernel} \label{fig:ta}
\end{subfigure}
\\
\vspace{0.2cm}
\\
\begin{subfigure}{\textwidth}
    \centering
         \begin{tabular}{|c | c | c | c | c |}
         \hline
         Label & MNIST & CIFAR-10 & Synthetic 1 & Synthetic 2\\ [0.5ex]
         \hline
         Original & 3 & 5 & 1 & 5 \\
         \hline
         Random & 7 & 5 & 4 & 4 \\
         \hline
         \end{tabular}
\caption{Laplacian Kernel} \label{fig:tb}
\end{subfigure}
\caption{Number of Epochs until training reached zero classification error.} \label{fig:table}
\end{figure}

In Figure \ref{fig:stopping} the Laplacian Kernel strongly outperforms the Gaussian in terms of test error, and that follows a general trend visible in all our experiments. Also very interesting in that regard is the high computational reach of the Laplacian Kernel (confer to Figure \ref{fig:table}), which is similar to the findings of Zhang et al. \cite{zhang2017} using neural networks with ReLU units. Therefore the Laplacian Kernel is not only better in terms of performance in our experiments, but also vastly superior in terms of computational effort.



\section{Bandwidth}

\begin{figure}[!htb]
\centering
\begin{subfigure}{0.42\textwidth}
% \includegraphics[width=\linewidth]{img/Figure8_Gaussian.png}
\caption{Gauss} \label{fig:bsa}
\end{subfigure}
\hspace*{\fill} % separation between the subfigures
\begin{subfigure}{0.42\textwidth}
% \includegraphics[width=\linewidth]{img/Figure8_Laplace.png}
\caption{Laplace} \label{fig:bb}
\end{subfigure}
\caption{Comparison between gauss, laplace.} \label{fig:bandwidth}
\end{figure}

Another interesting experiment can be seen in Figure \ref{fig:bandwidth}. Here we can see that smaller bandwidth (i.e. less smoothness in case of the Gaussian Kernel) results in better performance for higher levels of noise.  This gives us the intuition that maybe the possibility for sharper "cut outs" helps generalizing on noisy datasets.



\end{document}
