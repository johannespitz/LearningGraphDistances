\documentclass[a4paper,10pt]{article}
\usepackage[utf8]{inputenc}
\usepackage[margin=3cm]{geometry}
\usepackage{amsmath}
\usepackage{amssymb}
\usepackage{amsthm}
\usepackage{fancyhdr}
\usepackage{seminar}
\usepackage{graphicx}
% \usepackage{subfigure}
\usepackage{float}
\usepackage{hyperref}

\usepackage[round]{natbib}   % omit 'round' option if you prefer square brackets

\usepackage{mathrsfs}
\usepackage{commath}
\usepackage{mathtools}
\usepackage{subcaption}
\usepackage{appendix}

\DeclareMathOperator*{\argmax}{arg\,max}
\DeclareMathOperator*{\argmin}{arg\,min}

\pagestyle{fancy}


%You can add theorem-like environments (e.g. remark, definition, ...) if you want
\newtheorem{theorem}{Theorem}

\title{Learning graph distances} % Replace with your title
% \newcommand{\shorttitle}{\title}
% \shorttitle{Learning graph distances}
\author{Johannes Pitz} % Replace with your name
\institute{\textit{Guided Research: Data Analytics and Machine Learning Group  \protect\\ TUM Department of Informatics}}

\makeatletter
\let\runauthor\@author
\let\runtitle\@title
\makeatother
\lhead{\runauthor}
\rhead{\runtitle}


\begin{document}

\title{Learning graph distances \protect\\ via graph neural networks and node matching}
\maketitle

\begin{abstract}

Graph matching can be improved by proper application of sinkhorn, and graph distance learning can scale to large graphs by using matching.


\end{abstract}

\section{Introduction}

% Graphs are intersting because you can represent anything as a graph.
% Convolutional networks were extremly successful \cite{alexnet2012}.
% Graph Convolution Network, Kipf and Welling \cite{kipf2017}.
% Most research involving deep learning and graphs has focused on node and graph classification, but recently researchers have started applying these tools to the problem of graph distance estimation.

Most conceivable datasets can be represented as a graph or a set of graphs. Not only “graph-like” datasets such as social networks or chemical molecules fall into this category, also, i.e. executable binaries can be represented by control flow graphs or pictures can be represented by the graph of keypoints. % write better in more sentences

The broad range of applicability (of graphs), and the recent success of deep learning across different tasks [vision, rl, nlp] has spurred on the research in the application of deep neural networks to graphs. Borowing the idea of convolutional layers (\citealp{alexnet2012}) from conputer vision, \cite{kipf2017} introduced Graph Convolutional Networks (GCN) which can be implemented efficiently via message passing between connected nodes. Based on this message passing mechanism researchers have improved upon classical state-of-the-art methods in node classification and graph classification [find something in the pytorch-geometric list].

In this report we investigate the two problems:

\textbf{Graph Matching.} A graph matching establishes a node correspondence between graphs, maximising corrosponding node and edge affinity (\citealp{fey2020_update}; \citealp{wang2019}). For a detailed review of classical research into graph matching, recent neural network methods, and the applications of graph matching we refer the reader to \cite{fey2020_update}.
% Interesting applications here refer to Fey later maybe right before that paragraph

\textbf{Graph Distance/Similarity.} Graph distance or grpah similarity estimation refers to the problem of learning a metric relating different graphs from a given distribution. Intersting applications include detecting fraudulent binaries by means of control flow graphs (\citealp{{li2019}}), graph-based keyword spotting {\citealp{riba2018}}, and in particular predicting the Graph Edit Distance.
% but want to be able to learn any metric

The GED is widly used in graph similarity search, i.e. with reduced graphs of chemical compounds (\citealp{chem2006}), in fingerprint matching (\citealp{fingerprint2005}), or image indexing (\citealp{image_index2008}). Since computing the GED is NP-complete \cite{np_complete1998} it usually needs to be approximated.

Since computing the GED is NP-complete \cite{np_complete1998} it usually needs to be approximated. There are multiple classical algorithms to approximate the GED, some of them can gurantuee to find a lower bound (\cite{hungarian2009}) or upper bound (\citealp{hed2015}). However, due to their origin in optimization there is always a trade off between speed and accuracy, in which the faster approximations consider only very local node structures (\citealp{riba2018}).

% Faced with the great significance yet huge difficulty of computing the exact GED between two graphs,
% a flurry of approximate algorithms have been proposed with a trade-off between speed and accuracy.
% However, these methods usually require rather complicated design and implementation based on discrete optimization or combinatorial search. The time complexity is usually polynomial or even sub-exponential in
% the number of nodes in the graphs, such as HED
% (Fischer et al. 2015), Hungarian (Riesen and Bunke 2009),
% VJ (Fankhauser, Riesen, and Bunke 2011), A*-Beamsearch
% (Beam) (Neuhaus, Riesen, and Bunke 2006), etc

Previoius attemps of using neural networks for this have the short come that they cannot scale to larger graphs due to bottleneks (embedding vectors), or they (Riba) don't use Danskin.

We present a fast method to one using the other.

Main contributions:
\begin{itemize}
    \itemsep0em
    \item Propose method for graph distance learing that scales to larger graphs. \\ (With Nystrom/Multiscale $\rightarrow$ First sub $n^2$ that scales to larger graphs).
    \item Experimentaly show that it outperforms other methods on task of Graph Edit Distance estimation.
    \item Show that our application of sinkhorn has advantages over previous work.
\end{itemize}


\section{Related Work}

Graphs are intersting because you can represent anything as a graph.
Convolutional networks were extremly successful \cite{alexnet2012}.
Graph Convolution Network, Kipf and Welling \cite{kipf2017}.
Most research involving deep learning and graphs has focused on node and graph classification, but recently researchers have started applying these tools to the problem of graph distance estimation.

Riba et al. \cite{riba2018} proposed a siamese network consistis of two GCN and then Hausdorf Distance (chamfer).

Bai et al. \cite{bai2019} proposed to use attention to aggregate the embeddings and addtionally use histogram of pairwise distances (though not backpropable).
CNN1 \cite{bai2018_cnn1} \cite{bai2018_cnn2} requires somewhat arbitrary ordering.

Li et al. \cite{li2019} proposed to use allow messages passed from one graph to the other via cross-graph matching vector.

Fey at al. \cite{fey2020_update} propsed to use sinkhorn matching after GCN (and consensus stuff, which is not relevant to our work)


\section{Approach}

\subsection{Experiment 1}

At first we will train kernel classifiers on MNIST and CIFAR-10 using two different approaches. One will be called \textit{interpolated} and the other \textit{overfitted}.

Let $K(\mathbf{x}, \mathbf{z}) : \mathbb{R}^d \times \mathbb{R}^d \rightarrow \mathbb{R}$ be positive definite. Given data $\{(\mathbf{x}_i, y_i), i=1,...,n\}, \mathbf{x}_i \in \mathbb{R}^d, y_i \in \mathbb{R}$, then there exists a Reproducing Kernel Hilbert Space (RKHS) $\mathscr{H}$ of functions $f$ on $\mathbb{R}^d$. Moreover let $\mathbf{K}$ be the kernel matrix with $\mathbf{K}_{ij} = K(x_i, x_j)$. Then we are looking for the function $f^*$ which correctly classifies all training samples and minimizes the RKHS norm.

\begin{equation}
    f(\cdot) = \sum_{i=1}^{n} \alpha_{i} K(x_{i} ,\cdot)
\end{equation}
\begin{equation}
    f^* = \argmin_{f \in \mathscr{H}, f(\mathbf{x}_i)=y_i} \norm{f}_\mathscr{H} \text{, where } \norm{f}_\mathscr{H}^2 = \langle \boldsymbol{\alpha}, \mathbf{K} \boldsymbol{\alpha} \rangle = \sum_{ij} \alpha_i \mathbf{K}_{ij} \alpha_j
\end{equation}




\textbf{Interpolated:} By the classical representer theorem we get an explicit form to solve for the optimal $\alpha_i$.
\begin{equation}
    \boldsymbol{\alpha^*} =  \mathbf{K}^{-1} \mathbf{y}
\end{equation}
Note when we plug in the training data $X$ into the function $f^*$ we can see that indeed all samples are classified correctly.
\begin{equation}
\begin{aligned}
    f^*(\mathbf{X}) &= \mathbf{K} \boldsymbol{\alpha^*} \\
    &= \mathbf{K} \mathbf{K}^{-1} \mathbf{y} \\
    % &= \mathbf{y} \\
    f^*(\mathbf{x}_i) &= y_i \\
\end{aligned}
\end{equation}

\textbf{Overfitted:} To avoid having to solve the linear system which is prohibitive for larger datasets we can also pick any non-negative and strictly convex loss function $l$, such that $l(y, y) = 0$, and solve the unconstrained optimization problem.
\begin{equation}
    \boldsymbol{\alpha^*} = \argmin_{\boldsymbol{\alpha} \in \mathbb{R}} \sum_{j=1}^{N} l(\sum_{i=1}^{N} \alpha_{i} K(x_{i} ,x_{j}), y_i)
\end{equation}
Since $f^*$ minimizes $\sum_{j=1}^{N} l(f(x_i), y_i)$, this allows us to use gradient descent methods, and therefore shows the resemblance to a neural network. Basically, the weights between the input and the first layer are fixed (all the features of all training samples). This first layer has $N$ neurons, where $N$ is the number of training samples. The $\alpha_i$ are then the trainable weights connecting the first layer with the output layer.

\subsection{Experiment 2}

For the second experiment we will additionally use a synthetic datasets, and add label noise to both synthetic and real datasets.


\textbf{Adding noise:} We will randomly flip the true label of a $\epsilon$-fraction of the training and test samples to any of the possible labels with equal probability. Note that the Bayes Optimal Classifier remains the same on this new dataset and that the error rate scales linearly in $\epsilon$ between the error rate on the original dataset and random guessing (confer to Proportion 1 in Belkin et al. \cite{belkin2018a} for more details).

\textbf{Synthetic dataset:} The synthetic dataset has features $x \in \mathbb{R}^{50}$ and labels $y \in \{0, 1\}$. While $x_2 \sim \mathcal{U}(-1, 1), ..., x_{50} \sim \mathcal{U}(-1, 1)$ are drawn from the same uniform distribution, $x_1$ is drawn from a normal distribution depending on the class label.

\begin{equation}
x_1 \in \left\{
\begin{array}{@{}ll@{}}
    \mathcal{N}(0, 1), & \text{if}\ y=1 \\
    \mathcal{N}(2, 1), & \text{otherwise}
\end{array}\right.
\end{equation}

Note that the error of the Bayes Optimal Classifier is around $15.9\%$ for this dataset (the classes are not separable).


\subsection{General}

Throughout this report we will use two different kernels, the smooth Gaussain kernel $K(x,y) = \exp \left( - \frac{\norm{x-y}^2}{2\sigma^2} \right)$ and the non-smooth Laplacian kernel $K(x,y) = \exp \left( - \frac{\norm{x-y}}{\sigma} \right)$. Most of the overfitted examples are trained using the iterative EigenPro-SGD method by Ma and Belkin \cite{ma2017}. EigenPro is a preconditioned gradient descent iteration especially designed for fast convergence in kernel learning. In one occasion we used the Pegasos optimizer \cite{pegasos} because the EigenPro optimizer consistently converged towards random guessing on the training data instead of overfitting it. Both optimizer are available in the Keras Deep Learning library. The code for all experiments can be found on github (\href{https://github.com/johannespitz/kernel-overfitting}{link}).


\section{Results}



% Discussion
Table \ref{tab:ex1-baselines} shows that our method clearly outperforms all baselines. Best results on both datasets were found with the no-BP cost matrix, MLP-norm, and small weight decay for regularization. Interestingly, the only difference in the configurations is the  Sinkhorn regularization constant $\lambda$, which is smaller for the synthetic graphs. However, somewhat concerning is the gap between the validation and test performance on the real-world dataset, which also appears with the model of \cite{bai2019}. This phenomenon will be discussed in the ablation study.


\begin{table}[htbp]
    \addtolength{\tabcolsep}{-1pt}
    \fontsize{9pt}{10.25pt}\selectfont
    \centering
    \renewcommand{\arraystretch}{1.2}
    \begin{tabular}{|l|c|c|c|c|}
        \hline
        \multirow{2}{*}{} & \multicolumn{2}{c|}{Pref-Attachment} & \multicolumn{2}{c|}{AIDS} \\ \hhline{|~|-|-|-|-|}
        & Val & Test & Val & Test \\ \hhline{|=|=|=|=|=|}
        \cite{riba2018} & $12.2 \pm 0.2$ & $12.1 \pm 0.6$ & $15.5 \pm 0.3$  & $15.6 \pm 0.3$ \\ \hline
        \cite{bai2019} & $7.7 \pm 1.0$ & $9.6 \pm 2.5$ & $4.2 \pm 0.3$ & $8.7 \pm 0.1$ \\ \hline
        \cite{li2019} & $5.5 \pm 0.1 $ & $7.8 \pm 0.3$ & $10.6 \pm 0.3$ & $11.7 \pm 0.9$ \\ \hline
        GDN & $4.2 \pm 0.1$ & $\boldsymbol{4.5 \pm 0.2}$ & $3.3 \pm 0.04$ & $\boldsymbol{6.2 \pm 1.0}$ \\ \hline
    \end{tabular}
    \caption{RMSE to ground truth GED with standard deviation across 3 runs.}
    \label{tab:ex1-baselines}
\end{table}



The ablation study in Table \ref{tab:ex1-ablation} indicates that:
\begin{itemize}
    \itemsep0em
    \item \textbf{The Euclidean norm is superior to the manhattan norm, and a trainable MLP can improve results even further.}
    \item \textbf{There comes no immediate benefit from the full-size cost matrix.} In all but one case the no-BP cost matrix is at least as good as the BP variant in validation and test performance. Even though intuitively it feels convenient to have the option to explicitly add or remove nodes, it may be the case that we can get the same result by substituting one node for another in the case of the GED. However, using the no-BP matrix could certainly lead to problems when attempting to learn "sparse matchings" where only a few nodes have a match in the other graph. Moreover, it is not surprising that the results of both variants are very similar because the Sinkhorn normalization will place a lot of weight onto the 0 entries in the cost matrix. It does that to a point where summing over the part of the rows (or columns depending on which graph has more nodes) which had 0 entries before normalization yields values close to 1. Therefore, the values that contribute to the distance come mostly from entries which also show up in our no-BP variant matrix.
    \item \textbf{The AIDS validation set does not properly represent the entire distribution.} The difference in validation and test performance on the synthetic graphs can be explained by the variation between consecutive epochs. Since we stop on the best validation result it is no surprise that the test result at that exact point is slightly worse on average. In contrast, the massive gap in performance, and the high standard deviation on the AIDS test set indicate a shift of distributions between the validation and the test set. Some runs generalized well from the training data to the validation set, but they consistently performed worse on the test set across all epochs. Interestingly, not all runs behaved that way. We also collected many runs where the test performance is close to validation performance\footnote{Often these runs had higher Sinkhorn regularization but we did not run further experiments to verify that impression.}. Apparently the manhattan norm is more susceptible to overfit in this manner than the MLP norm.
\end{itemize}


\begin{table}[htbp]
    \addtolength{\tabcolsep}{-1pt}
    \fontsize{9pt}{10.25pt}\selectfont
    \centering
    \renewcommand{\arraystretch}{1.2}
    \begin{tabular}{|c|c|c|c|c|c|}
        \hline
        \multicolumn{2}{|c|}{} & \multicolumn{2}{c|}{Pref-Attachment} & \multicolumn{2}{c|}{AIDS} \\ \hline
        BP & Norm & Val & Test & Val & Test \\ \hhline{|=|=|=|=|=|=|}
        \multirow{3}{*}{yes} & $p=1$ & $5.2 \pm 0.1$ & $6.4 \pm 0.7$ & $3.9 \pm 0.1$ & $17.1 \pm 10.9$ \\ \hhline{|~|-|-|-|-|-|}
        & $p=2$ & $4.8 \pm 0.1$ & $5.7 \pm 0.2$ & $3.9 \pm 0.2$ & $8.6 \pm 2.6$ \\ \hhline{|~|-|-|-|-|-|}
        & MLP & $4.5 \pm 0.1$ & $\boldsymbol{4.5 \pm 0.3}$ & $3.5 \pm 0.03$ & $\boldsymbol{4.7 \pm 0.1}$ \\ \hline
        \multirow{3}{*}{no}  & $p=1$ & $4.9 \pm 0.2$ & $5.8 \pm 0.2$ & $3.9 \pm 0.3$ & $16.2 \pm 6.0$ \\ \hhline{|~|-|-|-|-|-|}
        & $p=2$ & $4.8 \pm 0.1$ & $5.7 \pm 0.6$ & $3.7 \pm 0.2$ & $6.8 \pm 3.4$ \\ \hhline{|~|-|-|-|-|-|}
        & MLP & $4.2 \pm 0.1$ & $\boldsymbol{4.5 \pm 0.2}$ & $3.3 \pm 0.04$ & $6.2 \pm 1.0$ \\ \hline
    \end{tabular}
    \caption{Ablation Study: RMSE with standard deviation across 3 runs.}
    \label{tab:ex1-ablation}
\end{table}


% Note bai has those 115 / 54 mse's

% Best settings?
% 0.0005 weight decay (not 0!)
% big learning rate
% sinkhorn_reg=0.2



\section{Conclusion}

% cost matrix

We proposed to use a cost matrix, motived by bipartite graph matching, and an explicit soft matching between nodes to learn an accurate and scalable estimation of metrics on graphs. Our model has the same theoretical complexity $\mathcal{O}(N_1 N_2)$ as previous neural network methods designed for this task. In practice the model runs quickly because our implementation can efficiently handle large mini-batches of arbitrarily sized graphs making it well suited for modern GPU setups. We showed empirically that the model improves upon state-of-the-art methods in predicting the ubiquitous graph edit distance.

% Using matching for distance computation has great potential. Can compute matching fast and still be fully differentiable. Works great on GED.
% no reason why we couldn't apply it to other distances


Closely following the setup of \cite{fey2020_update}, we applied our model to graph matching. In particular, motivated by the Sinkhorn distance, we proposed to apply Sinkhorn scaling to the kernel of the cost matrix. We showed empirically that with little tuning of the regularization constant this model improves upon the best results of \cite{fey2020_update} without the consensus step.

% mention that full-BP can be applied to sparse matching


\newpage

\bibliographystyle{plainnat}
\bibliography{egbib}

% cite    -> Name (year)
% citealp -> Name, year

\newpage

\appendix
\appendixpage

In this report I summarize the work I did during my guided research. The main part of the report focuses on findings that are relevant to a potential paper submission, but here in the appendix I will add some information about the datasets we used, and some we did not use (for completeness).


\section{Graph Edit Distance Datasets}
\begin{table}[htbp]
    \addtolength{\tabcolsep}{-1pt}
    \fontsize{9pt}{10.25pt}\selectfont
    \centering
    \renewcommand{\arraystretch}{1.2}
    \begin{tabular}{|l|c|c|c|c|c|c|c|}
        \hline
        Dataset & Node Feat & Edge Feat & \#Train & \#Val/Test & Max Nodes & Avg Nodes & Avg Edges \\
        \hline
        Pref-Att & 6 & 4 & 144 & 48 & 30 & 20.6 & 75.4 \\ % edges mean 75.4
        \hline
        AIDS & 53 & 4 & 144 & 48 & 30 & 20.6 & 44.6 \\ % edges mean 44.6
        \hline
    \end{tabular}
    \caption{Datasets}
    \label{tab:ex1-data}
\end{table}

Note \#Train and \#Val/Test are the number of graphs. We train and validate/test on all possible pairs of graphs, except the pairs where a graph is paired with itself.

\section{Yeast Dataset}

The Yeast Dataset (\url{https://www3.nd.edu/~cone/MAGNA++/}), used by \cite{yeast2019}, contains the protein-protein interaction (PPI) network of yeast and noise versions thereof. The base network contains 1,004 proteins and 4,920 high-confidence interactions. The noisy versions contain additional low-confidence interactions.

The first obstacle with this dataset would have been the generation of train/val/test splits.  Moreover, we could not have compared our results directly with \cite{yeast2019} because they learn unsupervised. The second obstacle is that the dataset contains no node or edge features. That is mostly a problem because GNNs are particularly good at using node, and also edge features. Therefore, it is questionable if a  model based on GNNs would perform impressive enough. In the end we decided not to use this dataset, also because we discovered the paper by \cite{fey2020_update} who use plenty of graph matching datasets that fit our model much better.

\section{Control Flow Dataset}\

\cite{li2019} used control flow graphs to evaluate their model. They used metric learning (margin-loss with pairs, and triplets) to recognize a function compiled from different compliers as similar, and different functions as not similar. They didn't share their dataset of FFmpeg functions with us, but referred us to a github repository, which I then forked (\url{https://github.com/johannespitz/functionsimsearch/tree/master}) and used to generate the UnRAR datasets, which (at the time of writing) can be found at '/nfs/students/pitz/cfgs' on the file server. To generate the graphcollections I used the notebook 'cfgs.ipynb' in the graph-distance repository. Note: use the .pickle version because it is much faster than .npz.

Currently we have a few different options for train/val/test splits. The simplest split is to use train\_all/val/test, which is an 80/10/10 split in terms of functions (not graphs). The attraction list contains all pairs of graphs that are from the same function, and the repulsion list contains as many negative samples. The train\_all split is then further split in: Option1, train\_across/val\_across/test\_acorss. Here out of the n graphs of any function we pick one to be the test graph, and all pairs between it and the remaining n-1 graphs are placed in the attraction list. Of the n-1 we then draw a validation graph and again all pairs between it and the remaining n-2 graphs are placed in the attraction list. The attraction list of the train\_across split contains then all possible pairs of the remaining n-2 graphs. Option2, train12/val1/val2/test1/test2. Here we split train\_all as chunk/val2/test2 (70/15/15) in terms of graphs (not functions), while ensuring that at least one graph of each function is contained the chunk. We then generate all positive pairs, but for the chunk we split those again into train12/val1/test1 (80/10/10) ensuring that every graph is in train12 at least once. All repulsion list for both options contain as many negative samples as the corresponding attraction list. Triplets are only available with the simple  train\_all/val/test (80/10/10) split.

Our model can be trained with pairs or triplets as it is. I did not include any results with the dataset in this report because the baselines cannot be trained on it (without putting in additional work).

\section{Consensus Step}
\label{appendix:consensus}

The main difference between our model and \cite{fey2020_update} is that they normalize the cost matrix $C$ instead of kernel matrix $e^{-\frac{C}{\lambda}}$. Consequently, they report bad gradients with Sinkhorn normalization and use row-wise sofmax instead. For their cost matrix they only use the pairwise distances between nodes without padding with norms, which should not be a problem for tasks where each node in the first graph has a match in the second graph, but could be problematic in other settings. In Table \ref{tab:diff-models} we collected a complete list of the differences between both models:


\begin{table}[htbp]
    \addtolength{\tabcolsep}{-1pt}
    \fontsize{9pt}{10.25pt}\selectfont
    \centering
    \renewcommand{\arraystretch}{1.2}
    \begin{tabular}{|C{1cm}|c|c|}
        \hline
        & DGC & GDN \\
        \hline
        $C^0_{i,j} $ & $h_i^0 \cdot h_j^0$ & $\begin{cases}
            \norm{h_i^0 - h_j^0}_2 & \text{if}\ 1 \leq i \leq N_1,\ 1 \leq j \leq N_2\\
            \norm{\alpha h_i^0}_2 & \text{if}\ N_1 < i \leq N_2,\ 1 \leq j \leq N_2
         \end{cases}$ \\
        % \hline
        $C^l_{i,j} $ & $C^{l - 1}_{i,j} + \text{MLP}(h_i^l - h_j^l)$ & $
        \begin{cases}
            \beta^l C^0_{i,j} + (1 - \beta^l) \text{MLP}(h_i^l - h_j^l) & \text{if}\ 1 \leq i \leq N_1,\ 1 \leq j \leq N_2\\
            \gamma^l \gamma^l C^{l - 1}_{i,j}  & \text{if}\ N_1 < i \leq N_2,\ 1 \leq j \leq N_2
        \end{cases}
        $ \\
        % \hline
        $M^l $ & $\operatorname{softmax}(C^l)$ & $\operatorname{sinkhorn}\left(e^{-\frac{C^l * \text{reg}^l}{\lambda}}\right)$ \\
        \hline
    \end{tabular}
    \caption{Differences of the two models}
    \label{tab:diff-models}
\end{table}

Note1: All superscripts indicate the layer of the consensus step.
Note2: $\gamma^l \in \mathbb{R}$ is a trainable parameter initialized with 1.
Note3: \cite{fey2020_update} use only those pairs where every node in the source graph finds a match in the target graph (therefore: $N_1 < N_2$). And since we use the no-BP cost matrix we have $1 \leq i, j \leq N_2$

% $$ \begin{cases}
%     \beta^l C^0_{i,j} + (1 - \beta^l) \text{MLP}(h_i - h_j) & \text{if}\ 1 \leq i \leq N_1,\ 1 \leq j \leq N_2\\
%     \gamma^2 C^0_{i,j}  & \text{if}\ i > N_1,\ 1 \leq j \leq N_2\\
%     \gamma^2 C^0_{i,j} & \text{if}\ 1 \leq i \leq N_1,\ j > N_2
%  \end{cases}
%  $$

% \begin{alignat*}{3}
%      & && \quad \text{\cite{fey2020_update}} && \quad \text{Ours} \\
%      &C^0_{i,j} &&= h_i \cdot h_j &&= \norm{h_i - h_j}_2 \qquad \text{(fill with norms)} \\
%      &C^l_{i,j} &&= C^{l - 1}_{i,j} + \text{MLP}(h_i - h_j) &&= \beta_l C^0_{i,j} + (1 - \beta_l) \text{MLP}(h_i - h_j) \qquad \text{(keep old norms scaled by } \alpha^2 \text{)} \\
%      &M^l &&= \operatorname{softmax}(C^l) &&= \operatorname{sinkhorn}(e^{-\frac{C^l * \text{reg}}{\lambda}}) \\
% \end{alignat*}

% We also implemented a consensus step based on their work. We use instead of
% \\
% $C_0 = h_s \cdot h_t$\\
% $C_0 = \norm{h_s - h_t}_2$ fill with norms\\
% $C_l = C_{l - 1} + \text{MLP}(h_s - h_t)$\\
% $C_l = \beta_l C_0 + (1 - \beta_l) \text{MLP}(h_s - h_t)$ keep old norms scaled by $\alpha^2$\\
% $M_l = \operatorname{softmax}(C_l)$ \\
% $M_l = \operatorname{sinkhorn}(e^{-\frac{C_l * \text{reg}}{\lambda}}) \quad \text{reg} = \frac{\norm{C_l}_1}{\max(n1,n2)^2}$\\



%%%%%%%%%%%%%%%%%%%%%%%%%%%%%%%%%%%%%%%%%%%%%%%%%%%%%%%%%%%%%%%%%%%%%%%%%%%%%%%%%%%%%%%%%%%
%%%%%%%%%%%%%%%%%%%%%%%%%%%%%%%%%%%%%%%%%%%%%%%%%%%%%%%%%%%%%%%%%%%%%%%%%%%%%%%%%%%%%%%%%%%
%%%%%%%%%%%%%%%%%%%%%%%%%%%%%%%%%%%%%%%%%%%%%%%%%%%%%%%%%%%%%%%%%%%%%%%%%%%%%%%%%%%%%%%%%%%


% \begin{equation}
%     \begin{gathered}
%          h_i^\text{final} = [h_i^L ; \dots ; h_i^1] \in \mathbb{R}^{d_\text{final}}\\
%          C_{i,j} = \norm{h_i^\text{final} - h_j^\text{final}} \\
%          C_{i, \epsilon} = \norm{ \alpha h_i^\text{final}} \quad
%          C_{\epsilon, j} = \norm{ \alpha h_j^\text{final}} \\ i \in \{1 \dots N_1\}, j \in \{1 \dots N_2\}
%     \end{gathered}
% \end{equation}


% \begin{equation}
% C_\text{no-BP} \in \mathbb{R}^{\max({N_1, N_2}) \times \max({N_1, N_2})}
% \end{equation}

% \begin{equation}
%     \frac{\partial d}{\partial c_{i,j}} = m^*_{i,j}
% \end{equation}


% \begin{equation}
%     d(C) = \min_{M} \langle M, C \rangle_\mathrm{F} + \lambda \text{H}(M) \quad \text{s.t. } \sum_i m_{i,j} = \sum_j m_{i,j} = 1
% \end{equation}

% \begin{equation}
%     M^* = \text{Sinkhorn}(e^{-\frac{C}{\lambda}})
% \end{equation}




% \begin{table}[htbp]
%     \addtolength{\tabcolsep}{-1pt}
%     \fontsize{9pt}{10.25pt}\selectfont
%     \centering
%     \renewcommand{\arraystretch}{1.2}
%     \begin{tabular}{|l|c|c|c|c|}
%         \hline
%         \multirow{2}{*}{} & \multicolumn{2}{c|}{Pref-Attachment} & \multicolumn{2}{c|}{AIDS} \\ \hhline{|~|-|-|-|-|}
%         & Val & Test & Val & Test \\ \hhline{|=|=|=|=|=|}
%         Riba et al. (2018) & $12.2 \pm 0.2$ & $12.1 \pm 0.6$ & $15.5 \pm 0.3$  & $15.6 \pm 0.3$ \\ \hline
%         Bai et al. (2019) & $7.7 \pm 1.0$ & $9.6 \pm 2.5$ & $4.2 \pm 0.3$ & $8.7 \pm 0.1$ \\ \hline
%         Li et al. (2019) & $5.5 \pm 0.1 $ & $7.8 \pm 0.3$ & $10.6 \pm 0.3$ & $11.7 \pm 0.9$ \\ \hline
%         GDN & $4.2 \pm 0.1$ & $\boldsymbol{4.5 \pm 0.2}$ & $3.3 \pm 0.04$ & $\boldsymbol{6.2 \pm 1.0}$ \\ \hline
%     \end{tabular}
%     \caption{RMSE to ground truth GED with standard deviation across 3 runs.}
%     \label{tab:ex1-baselines}
% \end{table}


% \begin{table}[ht]
%     \centering
%    \begin{tabular}{|c|c|c|}
%         \hline
%         \textbf{Method}&\textbf{Consensus} &\textbf{Mean} \\
%         \hline
%         DGC & L=0 &67.3\\
%         \hline
%         GDN & L=0&70.5\\
%         \hline
%         DGC & L=10&72.8\\
%         \hline
%         GDN & L=10&70.0\\
%         \hline
%     \end{tabular}
%     \caption{Hits@1 (\%) on PASCALVOC}
%     \label{tab:ex2}
% \end{table}



\end{document}
