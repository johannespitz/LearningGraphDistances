\section{Derivation of the model}

% (assume E << V^2)
We want our model to be applicable to any set of graphs $\mathcal{G}$, where each graph $G = \{V, E\} \in \mathcal{G}$ with nodes $V$ and edges $E$ is drawn from a common distribution. The nodes and edges may have arbitrary features. Possible distributions include chemical molecules, control flow graphs, or preferential attachment graphs (\citealp{pref_att2002}).

The Graph Edit Distance (GED) between two graphs is defined analog to the String Edit Distance.
\begin{equation}
     \text{GED}(G_{1},G_{2}) = \min_{(e_{1},...,e_{k}) \in \mathcal{P}(g_{1},g_{2})} \sum_{i=1}^{k} c(e_{i})
\end{equation}
where $e_{i}$ are edit operations: edge/node addtion, removal, and substituion, $c(e)$ the cost of an edit operation, and $\mathcal{P}(G_{1},G_{2})$ the edit paths that transform $G_{1}$ into a graph isomorphic to $G_{2}$.

Note that the GED and any other proper metric on graphs is symmetric and invariant to the graph representation. Clearly the GED as defined above cannot depend on the ordering in which the nodes are saved in memory. Moreover, as long as the costs of substituion are symmetric, also the GED is symmetric ($\text{GED}(G_{1},G_{2}) = \text{GED}(G_{2},G_{1})$). Therefore, a neural network approximating the GED should also be symmetric and invariant to the graph representation.

A simple but flawed idea is to embed both graphs into a common vector space and compute the distance between these two vectors. One can use message passing networks to generate node embeddings. Message passing can be applied to graphs with arbitrary features; it is invariant to the graph representation and commom practice in machine learning on graphs due to efficient handling of sparse inputs. These node embeddings can then be aggregated into a single graph embedding, possibly using an attention layer. Finally, any distance between two vectors (cosine, $p$-norm, etc.) will ensure symmetric predictions. The main problem with this approach is that the embedding size becomes a bottleneck for larger graphs and the entire model would need to be trained from scratch to increase it.
%  \cite{li2019}, and \cite{bai2019}

To solve the problem we propose\footnote{\cite{riba2018} already used an implicit matching for predicting the GED with neural networks but we will improve on their work by utilzing modern matching tools.} to match the nodes of both graphs, similar to classical GED algorithms (\citealp{hungarian2009}; \citealp{frankhauser2011}). These algorithms are based on bipartite graph matching. They set up a cost Matrix $C \in \mathbb{R}^{N \times N}$, where $N = N_1 + N_2 = \vert V_1 \vert + \vert V_2 \vert $ and solve the following contrained optimization problem.

\vspace{.2cm}

\noindent
\begin{minipage}{.5\linewidth}

     \[
          C=
               \left[
               \begin{array}{ccc|ccc}
                    C_{1,1} & \dotsi & C_{1, N_2} & C_{1, \epsilon} & \dotsi & \infty \\
                    \vdots & \ddots & \vdots & \vdots & \ddots & \vdots \\
                    C_{N_1, 1} & \dotsi & C_{N_1, N_2} & \infty & \dotsi & C_{N_1, \epsilon} \\
                    \hline
                    C_{\epsilon, 1} & \dotsi & \infty & 0 & \dotsi & 0 \\
                    \vdots & \ddots & \vdots & \vdots & \ddots & \vdots \\
                    \infty & \dotsi & C_{\epsilon, N_2} & 0 & \dotsi & 0 \\
               \end{array}
               \right]
     \]

\end{minipage}%
\begin{minipage}{.5\linewidth}

     \begin{equation}
          \begin{gathered}
               % \min \sum_{i = 1}^{N} \sum_{j = 1}^{N} M_{ij} C_{ij} = d(M, C) \\
               d(M, C) = \min \norm{M * C}_F \\
               \text{subject to} \\
               \sum_{i = 1}^{N} M_{ij} = 1 \forall j \in \{1 \dots N\} \\
               \sum_{j = 1}^{N} M_{ij} = 1 \forall i \in \{1 \dots N\} \\
               M_{ij} \in \{0, 1\}
          \end{gathered}
     \end{equation}

\end{minipage}


\vspace{.2cm}

Here $*$ denotes the Hadamard product, and  $\norm{\cdot}_F $ the Frobeniusnorm. $C_{i, j}$ is the cost for replacing node $i$ with node $j$, $C_{\epsilon, j}$ the cost for inserting node $j$, and $C_{i, \epsilon}$ the cost for deleting node $i$. These costs depend on the direct neighborhood of the node, and can even be extend with random walks around it, but they lack accuracy in cases "where the direct neighbourhood of graphs is not suciently discriminant" (\citealp{hungarian2009}). The assignment problem can be solved exaclty by algorithms such as the Hungarian method, also called the Kuhn-Munkres algorithm. (\citealp{hungarian1955}) or the VJ algorithm (\citealp{vj1987}).

Since we want our model to be capable of learning any metric on graphs we use a message passing network to learn the entires of the cost matrix $C$. The network generates node embeddings that contain information about their local neighborhood. Instead of aggregating this information we compute pairwise distances between nodes of both graphs yielding a trainable equivalent to the upper left bock of the cost matrix. The corresponding values for insertion and deletion can be computed by taking a norm of the embeddings. Analog to the classical approach we then compute the distance as $ d = \norm{M * C}_F $.

For finding a matching there are multiple possiblities:
\begin{itemize}
     \item \textbf{Nearest Neighbor.} Taking the sum of the row-wise minima (and, for the sake of symmetry, the sum of the column-wise minima) yields what is known in computer vision as the \textbf{Chamfer Distance}. The matching runs in $O(N^2)$ and can be implemented using the $\argmax$ operator which can be backpropagted and is already implemented in modern deep learning frameworks.
     \item \textbf{Optimal Assignment.} Applying the Kuhn-Munkres algorithm results in an optimal assignment, $M_{ij} \in \{0,1\}$, in roughly $O(N^3)$ operations. However, this is a discrete algorithm, and therefore not differentiable.
     \item \textbf{Optimal Transport.} The optimal transport problem is a continous relaxation of the optimal assignment, where $M_{ij} \in \left[ 0,1 \right]$. In this setting usually called the \textbf{Earth Mover Distance} it can be found in roughly $O(N^3 \log(N))$ operations.
     \item \textbf{Regularaized Optimal Transport.} \cite{sinkhorn2013} introduced an algorithm for solving an entropy regularized version of the optimal transport problem minimizing $\norm{M * C}_F + \frac{1}{\lambda} H(M)$. One simply computes the kernel matrix $ K = e^{ -\frac{C}{\lambda}}$ and applies Sinkhorn’s iterative matrix scaling. This is called the \textbf{Sinkhorn Distance} and can be computed extremly fast in practice with worst case running time of $O(N^2)$.
\end{itemize}
While the chamfer distance can easily be backpropagated, the other three all minize a convex function. In particular, $d(M,C)$ is convex in $C$ for every $M$, and the set of all possible matchings $M$ is compact. Therefore, we can apply Danskin's theorem (\citealp{danskin1967}) to find the derivative of the distance $d$ with respect to the cost matrix $C$. The theorem states that the derivative of a funciton of the form $f(x) = \max_{z \in Z} \phi(x,z)$ is the derivative at the optimum, which, in this case, is simply the optimal matching $M$. That means we can backpropagte any of the distances mentioned above, and learn in end-to-end fashion.
% $f(x) = \max_{z \in Z} \phi(x,z)$

Preliminary experiments indicated that the sinkhorn distance is not only similar fast to compute as the chamfer distance, but also yields better results than the other much slower methods. The regularization results into softer matchings compared to hard assignments, or the still very peaky matchings of the earth mover's distance. We believe these soft matchings propagate more gradient information to the message passing network allowing it to thoroughly adapt to the given distribution of graphs. Note that by adjusting the regularization parameter $\lambda$ we can control how soft the matchings are.

% TODO
% add metric learning loss

% TODO
In response to Fey et al. we also evaluated our method on graph matching. When we need the matching we cannot apply Danskin's theorem, but sinkhorn at around 50 iterations can still be backpropagated just fine. We wanted to see how using the kernel as we do (and not simply scaling the cost matrix) would also improve upon their results before/after consensus.

% Note wikipedia gives best running times as $O(mn + n^2 * \log(n))$ for Munkres
% and $O(N \log(N)^2)$ for EMD
