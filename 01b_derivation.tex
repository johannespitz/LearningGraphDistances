\section{Derivation of the model}

% derivation
%     (cost_matrix -> matching) premutaiton invariant (matching is because invariant)

%     GED
%     compare two graphs

%     try matching

%     discrete -> non differentiable
%     -> EMD slow

%     sinkhorn

%     matrix scaling

%     backprop in optimal (Danskin)

%     (nystrom, multiscale sinkhorn)
%     (lanmarks, cluster)

%     GNN
%     Consensus



% (assume E << V^2)
Striving for a broad applicability of our model, let $\mathcal{G}$ be a set of graphs. Each graph $G = \{V, E\} \in \mathcal{G}$ with nodes $V$ and edges $E$ is drawn from a common distribution. The nodes and edges may have arbitrary features. Possible distributions include chemical molecules, control flow graphs generated by C++ compilers, or preferential attachment graphs (\citealp{pref_att2002}).

The Graph Edit Distance (GED) between two graphs is defined analog to the String Edit Distance.
\begin{equation}
     \text{GED}(G_{1},G_{2}) = \min_{(e_{1},...,e_{k}) \in \mathcal{P}(g_{1},g_{2})} \sum_{i=1}^{k} c(e_{i})
\end{equation}
where $e_{i}$ are edit operations: edge/node addtion, removal, and substituion, $c(e)$ the cost of an edit operation, and $\mathcal{P}(G_{1},G_{2})$ the edit paths that transform $G_{1}$ into a graph isomorphic to $g_{2}$.

Since computing the GED is NP-complete \cite{np_complete1998} it usually needs to be approximated. There are multiple classical algorithms to approximate the GED, some of them can gurantuee to find a lower bound (\cite{hungarian2009}) or upper bound (\citealp{hed2015}). However, due to their origin in optimization there is always a trade off between speed and accuracy, in which the faster approximations consider only very local node structures (\citealp{riba2018}).

% set up bp cost_matrix

%% From CNN2
% Exactly solving this constrained optimization
% program would yield the exact GED solution
% (Fankhauser, Riesen, and Bunke 2011), but it is NPcomplete since it is equivalent to finding an optimal matching in a complete bipartite graph (Riesen and Bunke 2009).
% To efficiently solve the assignment problem, the Hungarian algorithm (Kuhn 1955) and the Volgenant Jonker (VJ)
% (Jonker and Volgenant 1987) algorithm are commonly used,
% which both run in cubic time. In contrast, GSimCNN takes
% advantage of the exact solutions of the instances of this problem during the training stage, and computes the approximate
% GED during testing in quadratic time, without the need for
% solving any optimization problem for a new graph pair.

% message passing formulas
